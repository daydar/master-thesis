\documentclass[11pt,a4paper]{report} 

% Für doppelseitigen Ausdruck (nur bei > 60 Seiten sinnvoll)
% \usepackage{ifthen}
% \setboolean{@twoside}{true}
% \setboolean{@openright}{true} 

\include{preamble} % alle Pakete und Einstellungen

% Hier anpassen 
% \newcommand{\welchethesis}{Bachelor}
\newcommand{\welchethesis}{Master}
\newcommand{\thesisofwas}{of Science}
\newcommand{\titel}{Data Mining komplexer Datenstrukturen aus PDF-Dokumenten}
\newcommand{\kurztitel}{Data Mining PDF Dokumente}
\newcommand{\autor}{Deniz Aydar}
\newcommand{\datum}{25. Juli 2022} % Abgabedatum
\newcommand{\ort}{Wiesbaden}
\newcommand{\referent}{Prof.\ Dr.\ Dirk Krechel}
\newcommand{\korreferent}{Prof.\ Dr.\ Philipp Schaible}

\begin{document}

\begin{titlepage}
    \begin{center}
      % Kopf der Seite
      \hsrmlogo[1]
      \parbox[t]{8cm}{
        % \textsf würde das Aussehen der ersten Seite ruinieren, 
        % wer will, soll das selbst außen rum machen...
        Hochschule \textbf{RheinMain}\\
        Fachbereich Design Informatik Medien\\
        Studiengang Medieninformatik}
      \vfill    
      {\LARGE Abschlussarbeit} \\[0.5cm]
      {\large zur Erlangung des akademischen Grades} \\[5mm]
      {\large \welchethesis\ \thesisofwas} \\[5mm]
      \rule{\textwidth}{1pt}\\[0.5cm]
      {\begin{spacing}{1.15} \huge \bfseries \titel \\ \end{spacing}}
      \rule{\textwidth}{1pt}    
      \vfill    
      \begin{tabular}{ll} % Mitte der Seite
        Vorgelegt von & \autor \\
        am & \datum \\
        Referent & \referent \\
        Korreferent & \korreferent
      \end{tabular}    
      \vfill
    \end{center}
  \end{titlepage}
  \cleardoublepage
  
  
  % Erklärung gemäß den Allgemeinen Bestimmungen für Prüfungsordnungen
  % der Paragraph schwankt, daher ohne Nennung einer Nummer
  \thispagestyle{empty}
  \section*{Erklärung gemäß ABPO}
  Ich erkläre hiermit, dass ich
  \begin{itemize}
  \item die vorliegende Abschlussarbeit selbstständig angefertigt,
  \item keine anderen als die angegebenen Quellen benutzt,
  \item die wörtlich oder dem Inhalt nach aus fremden Arbeiten entnommenen 
    Stellen, bildlichen Darstellungen und dergleichen als solche genau 
    kenntlich gemacht und
  \item keine unerlaubte fremde Hilfe in Anspruch genommen habe.
  \end{itemize}
  
  \vspace{6em}
  \noindent\begin{tabular}{p{0.37\textwidth}p{0.56\textwidth}}
  \ort, \datum  & \rule{0.56\textwidth}{0.5pt}\\
                & \makebox[1cm]{\ } \autor
  \end{tabular}
  
  \vfill
  
  \section*{Erklärung zur Verwendung der \welchethesis thesis}
  
  Hiermit erkläre ich mein Einverständnis mit den im folgenden 
  aufgeführten Verbreitungsformen dieser Abschlussarbeit:
  
  \vspace{1em}
  \noindent\begin{tabular}{|p{0.82\textwidth}|c|c|}
    \hline
    \textbf{Verbreitungsform} & \makebox[0.035\textwidth]{\textbf{Ja}} 
                              & \makebox[0.05\textwidth]{\textbf{Nein}} \\\hline
    Einstellung der Arbeit in die Hochschulbibliothek 
                           mit Datenträger   &  & $\times$ \\\hline
    Einstellung der Arbeit in die Hochschulbibliothek  
                           ohne Datenträger  &  & $\times$  \\\hline
    Veröffentlichung des Titels der Arbeit im Internet  
                                             & $\times$ & \\\hline
    Veröffentlichung der Arbeit im Internet             
                                             &  & $\times$ \\\hline
  \end{tabular}
  
  \vspace{6em}
  \noindent\begin{tabular}{p{0.37\textwidth}p{0.56\textwidth}}
  \ort, \datum  & \rule{0.56\textwidth}{0.5pt}\\
                & \makebox[1cm]{\ } \autor
  \end{tabular}
  \cleardoublepage
  
   % Titelseite, Erklärungen, etc.

\begin{abstract} 
PDF-Dokumente besitzen viele Informationen aus denen sich neue Daten generieren lassen können.
Doch das Extrahieren von solchen Daten ist heutzutage immer noch mit Hürden verbunden.
Dies gilt auch für die Dokumente die in den Prozessen von Autohäusern und Kfz-Werkstätten verwendet 
und anschließend gelagert werden.
Um die Frage zu beantworten inwiefern neue Daten aus diese Art von Dokumenten verarbeiten lassen, 
wird im Rahmen dieser Arbeit ein System konzipiert und entwickelt welches es ermöglichen soll weiteres Wissen zu gestalten.

\end{abstract}

\tableofcontents


\chapter{Einleitung} \label{chap:einf}

„Digitalisierung im Alltag voranbringen“ – Das war einer der Wahlslogans während der Bundestagswahl 2021. 
Gemeint war damit eine zunehmende Digitalisierung im privaten Alltag vieler Bürger:innen, 
aber auch in der Wirtschaft machte sich wachsend der Wunsch nach mehr digitalen Alternativen breit (vgl. Bundesregierung 2021). 
Dieser Wunsch beinhaltete vor allem einen Wechsel von gängigen Papierformen verschiedenster Dokumente 
zu denselben in digitaler Ausprägung. 

Genau jenes Bedürfnis nach Digitalisierung betrifft auch Arbeitnehmer:innen / Händler:innen in Autohäusern und Kfz-Werkstätten, 
die durch ihre berufliche Tätigkeiten mit einer Vielzahl an Unterlagen, 
Dokumenten oder Belegen arbeiten müssen. Unter dieser Vielzahl fallen Dokumente wie Werkstatt-, Kauf- und Mietverträge 
sowie Rechnungen oder Diagnoseberichte. 

Durch eine Digitalisierung jener Dokumente eröffnen sich Vorteile wie bessere Zugänglichkeit, 
größere Langlebigkeit und vor allem eine angepasste und leichtere Nutzung, 
die zu einer höheren Effektivität innerhalb täglicher Arbeitsschritte führt.
Durch bisherige erste Schritte der Digitalisierung sind diese benötigten Dokumente 
bereits elektronisch aufgearbeitet und den Arbeitnehmenden in Autohäusern und Kfz-Werkstätten zur Verfügung gestellt worden. 

Die Folge dessen ist, dass alle Dokumente standardisiert sind und dadurch die in den Dokumenten beinhalteten Informationen 
neu verarbeitet werden können. Eine Extraktion der Daten der einzelnen Dokumente ist jedoch nur eingeschränkt möglich, 
da die Struktur im gängigen Gebrauch im Dateiformat PDF (Portable Document Format) festgelegt ist. 

Jene Limitierung der Datenextraktion wirkt sich gleichermaßen auf die Dealer-Management-Systeme (DMS), 
mit solchen die Arbeitnehmenden ihr Prozesse abwickeln, aus. Der daraus resultierende Arbeitsaufwand, 
welcher dabei entsteht, um die Daten wiederverwendbar zu konstruieren, ist derzeit enorm. 

\section{Zielsetzung}\label{sec:ziel}

Aus diesem Grund möchte ich innerhalb dieser Master-Thesis ein System entwickeln, 
das genau jene Problematik erleichtert und löst.
Innerhalb üblicher Methoden werden heutzutage die Daten zunächst über eine grafische Anzeige mithilfe 
einer bereits existierenden Benutzeranwendung via gängiges Kopieren und Einfügen entnommen. 
Dieser Schritt funktioniert zwar im Regelfall, ist jedoch oftmals stark fehleranfällig und kann lückenhaft sein. 

In so einem Fall muss das Einfügen und Kopieren manuell durchgeführt werden, was bei einer riesigen Menge an Dokumente zu einer monotonen Arbeit führt. 
Andernfalls können die Daten eigenhändig abgeschrieben werden – jener Vorgang benötigt jedoch viele Ressourcen.
Eher geeignet ist es, einen Datenkonverter für die PDF Dateien zu nutzen, um so die Dokumente in Bilder beispielsweise umzuwandeln, 
wodurch die Daten zugänglicher sind, aber immer noch verarbeitet werden müssen, was den gesamten Vorgang als eine Notlösung wirken lässt.

Aufgrund dieser bisherigen teils aufwendigen und mit Fehlern verbundenen Möglichkeiten möchte ich mich in meiner Thesis 
von diesen Optionen abwenden und das Data-Mining nutzen. 
Das Data-Mining soll eine Automatisierung unterstützen, mit der Daten aus dem PDF-Dokument extrahiert werden. 
Allein über das Data-Mining ist es möglich, Inhalte zu extrahieren und für andere Systeme bereit zu stellen. 

Der Unterschied zum gängigen Data-Mining liegt dabei in den Einschränkungen des Dateiformats: 
der Quelltext einer solchen Datei stellt erstens keine klare Hierarchie der Daten dar und zweitens 
besitzt es durch das Fehlen von Markierungen keine Informationen darüber, 
was die Daten an sich darstellen sollen \cite{docsumo_pdf_2022}. 

Mit Hilfe dieser Technik des Data-Minings soll es ermöglicht werden, aktuelle Prozesse detaillierter zu steuern und zu überwachen. 
Des Weiteren können unter anderem Abgleiche von Rechnungen mit dem System durchgeführt werden 
und weitere Datenanalyse und Reporting kreiert werden.
Mit dem derzeitigen Stand der Datenextraktion können jedoch lediglich Agierende aus dem technischen Bereich arbeiten. 
Die Technologien hierfür benutzen unterschiedliche Ansätze und müssen daher zunächst für diesen Fall ausgewählt werden. 

Damit jedoch auch für Agierende innerhalb des technischen Bereichs die Nutzung nicht zu abstrakt bleibt, 
soll durch die Entwicklung eines Systems der Zugriff greifbarer gestaltet werden, welches wiederum eine benutzerfreundliche Anwendung zur Verfügung stellen soll. 
Darüber hinaus muss eine Automatisierung vorhanden sein, um auch eine große Datenmenge verarbeiten zu können, 
und die Vorgänge sollen außerdem eine niedrigere Fehlerquote aufzeigen. 
Die Prozesse müssten außerdem das ganze über eine Steuerung und Regelung kontrollieren.
Solche Prozesse können zum Beispiel über Regeln festgelegt werden, die bei Erfüllung weitere Aktionen auslösen können.

Eine solche Eigenschaft kann durch eine symbolische künstlichen Intelligenz (KI) abgedeckt werden, welche eine vorgegebene Verarbeitung möglich macht. 
Hiermit bezeichnet man einen altmodischen Ansatz der KI, bei der über das Festlegen von Symbolen 
menschliches Wissen in einer Logik gehalten wird und diese benutzt wird um weiteres Wissen zu generieren \cite{dickson_what_2019}.
Durch eine Parametrisierung einer solchen KI kann dann die Unschärfe der ausgewählten Daten festgelegt werden, 
damit eine Feineinstellung stattfinden kann. 
Das heißt, der Benutzer eines solchen Systems soll sich beginnend von groben Definitionen für die Verarbeitung 
zu einer detaillierten und angepassteren Definition über Feedback des Systems hinarbeiten. 

Mit Hilfe dieser Annäherung an Definitionen und Regeln kann so für eine bessere Erfolgsquote gesorgt werden. 
Zudem kann der Ansatz des Machine Learnings (ML), wobei erfolgreiche Verarbeitungen einem System antrainiert werden, 
weitere Dokumente aus Daten extrahieren. 
Hierbei ist noch offen, welcher Typ des Machine Learnings sich für diesen Fall eignet.

Beide Ansätze – der der symbolischen KI und der des Machine Learnings – bieten als Option die Entwicklung einer grafischen Entwicklungsumgebung (IDE) an. 
Jene soll das Festlegen von Definitionen und Regeln erlauben, mit welchen die Dokumente verarbeitet werden können. 
Außerdem soll diese Oberfläche verschiedene Funktionalitäten anbieten und auch Feedback sowohl bei einem problemlosen Lauf 
als auch bei einem fehlerhaften Lauf zurückgeben. 

Des Weiteren soll das System es ermöglichen Änderungen der Definitionen anzumerken und Unterschiede zu erstellen, 
womit auch bisherige Ergebnisse angezeigt werden.
Die Festlegung von Definitionen und Regeln soll außerdem durch eine Ansicht der Dokumente unterstützt werden.
Insgesamt wird das System die zwei Ansätze als Subsysteme aufteilen um die Umsetzung zu modularisieren und einen Vergleich zu ermöglichen.

\section{Ablauf}\label{sec:ablauf}

Um genau dieses optimale System für das beschriebene Szenario entwickeln zu können, werde ich in dieser Arbeit wie folgt vorgehen: 

Für eine vollständige Auseinandersetzung soll eine weitere Analyse stattfinden,
damit die vorhandenen Grundlagen für den Anwendungsfall evaluiert werden können. 
Das heißt, es werden mit Hilfe der Funktionalitäten der Bibliotheken Ergebnisse 
auf Basis der Datensätze erzeugt, um die Erfolgsquoten zu überprüfen. 
Diese werden dann verglichen und ausgewertet. Parallel dazu soll die Umgebung 
für die Entwicklung eingerichtet werden, mit der die Implementierung der Systeme stattfinden soll. 

Danach widme ich mich der Konzeption und der Entwicklung des Systems.
Dies beginnt mit der symbolischen KI als ein Subsystem des gesamtem Projekts, sodass ein direkterer Ansatz ermöglicht wird. 
Die Entwicklung hierbei wird im Wasserfall-Ansatz, einem Ansatz bei der phasenweise die Eigenschaften einer solchen KI umgesetzt werden, 
um eine Grundlage für die nächste Komponente zu bieten.

Bei dieser Komponente handelt es sich um die IDE, die den Zugang für den Benutzer zum System darstellt.
Deswegen folgt im nächsten Schritt die Konzeptionierung und Entwicklung der IDE, 
welche sich während der Entwicklung der symbolischen KI parallelisieren lässt. 
Sobald das Gerüst der Benutzeroberfläche fertiggestellt ist, möchte ich dies mit der symbolischen KI verbinden.

Als Gegenstück wird sich dann mit dem Machine Learning Subsystem, bei der auch eine Konzeptionierung und Entwicklung, gewidmet.
Hier wird das ganze in einem iterativen Ansatz umgesetzt um so den Hauptteil so früh wie möglich parat zu haben, 
damit das Training des ML-Systems durch die Dokumente auch zeitnah starten kann. 
Durch das Auswerten dieses Subsystems wird es daraufhin angepasst. 

Danach wird das Testing für beide Subsysteme eingeführt um so das erfolgreiche Ausführen gewisser Eigenschaften abzudecken.
Dies dient wiederum als Grundlage um die Continuous Integration (CI) für das System einzuführen, 
womit eine sichere und konsistente Entwicklung nach dem Prototyping angeboten wird.   

Zum Schluss werden sich die Ergebnisse des gesamten Systems betrachtet und ein Ausblick zu der Thematik dargestellt.

\chapter{Hintergrund}\label{chap:hint}

Um die Problematik detaillierter darzustellen und um eine Übersicht zu ermöglichen, 
werden in diesem Kapitel der Rahmen der Problematik aus technischer Sicht vertieft dargestellt.
Hierbei sollen die Technologien abgewogen und evaluiert werden. 

\section{Analyse}\label{sec:ana}

Die 


\chapter{Konzeption und Umsetzung} \label{chap:konundum} %KAPITEL

Für den Sprung zur Umsetzung bedarf es als Nächstes die Konzeption des technischen Rahmens. 

\chapter{Ergebnisse} \label{chap:erg}



\chapter{Ausblick und Fazit} \label{chap:auf}
 
Im Rahmen der Abschlussarbeit wurde 
 technische Funktionalitäten eingeführt werden um dem Kapitän die Webanwendung als PWA anzubieten.

Insgesamt eignen sich Webanwendungen um Ressourcen übersichtlicher darzustellen und mobil zu beobachten.
Auch können Offline-Zustände dem Benutzer trotzdem erlauben Aspekte der Anwendung weiter zu benutzen.
Aus den Konzepten und der Umsetzung ergeben sich viele weitere Ideen und Ansätze, die realisierbar sind und implementiert werden können.  

\newpage

% Listen wenn überhaupt ans Ende und nicht an den Anfang.
% Meist ist das aber unnötig.
%\listoffigures % Liste der Abbildungen 
%\listoftables % Liste der Tabellen
% \newpage

\bibliography{thesis}
\bibliography{online}
\bibliographystyle{plain} % Literaturverzeichnis
\begin{btSect}{thesis} % mit bibtopic Quellen trennen
\section*{Literaturverzeichnis}
\btPrintCited
\end{btSect}
\begin{btSect}{online}
\section*{Online-Quellen}
\btPrintCited
\end{btSect}
% dann mit "bibtex thesis1" und "bibtex thesis2" arbeiten

\end{document}
;;; Local Variables:
;;; ispell-local-dictionary: "de_DE-neu"
;;; End:
