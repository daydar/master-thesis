\documentclass[11pt,a4paper]{report} 

% Für doppelseitigen Ausdruck (nur bei > 60 Seiten sinnvoll)
% \usepackage{ifthen}
% \setboolean{@twoside}{true}
% \setboolean{@openright}{true} 

\include{preamble} % alle Pakete und Einstellungen

% Hier anpassen 
% \newcommand{\welchethesis}{Bachelor}
\newcommand{\welchethesis}{Master}
\newcommand{\thesisofwas}{of Science}
\newcommand{\titel}{Data Mining komplexer Datenstrukturen aus PDF-Dokumenten}
\newcommand{\kurztitel}{Data Mining PDF Dokumente}
\newcommand{\autor}{Deniz Aydar}
\newcommand{\datum}{25. Juli 2022} % Abgabedatum
\newcommand{\ort}{Wiesbaden}
\newcommand{\referent}{Prof.\ Dr.\ Dirk Krechel}
\newcommand{\korreferent}{Prof.\ Dr.\ Philipp Schaible}

\begin{document}

\begin{titlepage}
    \begin{center}
      % Kopf der Seite
      \hsrmlogo[1]
      \parbox[t]{8cm}{
        % \textsf würde das Aussehen der ersten Seite ruinieren, 
        % wer will, soll das selbst außen rum machen...
        Hochschule \textbf{RheinMain}\\
        Fachbereich Design Informatik Medien\\
        Studiengang Medieninformatik}
      \vfill    
      {\LARGE Abschlussarbeit} \\[0.5cm]
      {\large zur Erlangung des akademischen Grades} \\[5mm]
      {\large \welchethesis\ \thesisofwas} \\[5mm]
      \rule{\textwidth}{1pt}\\[0.5cm]
      {\begin{spacing}{1.15} \huge \bfseries \titel \\ \end{spacing}}
      \rule{\textwidth}{1pt}    
      \vfill    
      \begin{tabular}{ll} % Mitte der Seite
        Vorgelegt von & \autor \\
        am & \datum \\
        Referent & \referent \\
        Korreferent & \korreferent
      \end{tabular}    
      \vfill
    \end{center}
  \end{titlepage}
  \cleardoublepage
  
  
  % Erklärung gemäß den Allgemeinen Bestimmungen für Prüfungsordnungen
  % der Paragraph schwankt, daher ohne Nennung einer Nummer
  \thispagestyle{empty}
  \section*{Erklärung gemäß ABPO}
  Ich erkläre hiermit, dass ich
  \begin{itemize}
  \item die vorliegende Abschlussarbeit selbstständig angefertigt,
  \item keine anderen als die angegebenen Quellen benutzt,
  \item die wörtlich oder dem Inhalt nach aus fremden Arbeiten entnommenen 
    Stellen, bildlichen Darstellungen und dergleichen als solche genau 
    kenntlich gemacht und
  \item keine unerlaubte fremde Hilfe in Anspruch genommen habe.
  \end{itemize}
  
  \vspace{6em}
  \noindent\begin{tabular}{p{0.37\textwidth}p{0.56\textwidth}}
  \ort, \datum  & \rule{0.56\textwidth}{0.5pt}\\
                & \makebox[1cm]{\ } \autor
  \end{tabular}
  
  \vfill
  
  \section*{Erklärung zur Verwendung der \welchethesis thesis}
  
  Hiermit erkläre ich mein Einverständnis mit den im folgenden 
  aufgeführten Verbreitungsformen dieser Abschlussarbeit:
  
  \vspace{1em}
  \noindent\begin{tabular}{|p{0.82\textwidth}|c|c|}
    \hline
    \textbf{Verbreitungsform} & \makebox[0.035\textwidth]{\textbf{Ja}} 
                              & \makebox[0.05\textwidth]{\textbf{Nein}} \\\hline
    Einstellung der Arbeit in die Hochschulbibliothek 
                           mit Datenträger   &  & $\times$ \\\hline
    Einstellung der Arbeit in die Hochschulbibliothek  
                           ohne Datenträger  &  & $\times$  \\\hline
    Veröffentlichung des Titels der Arbeit im Internet  
                                             & $\times$ & \\\hline
    Veröffentlichung der Arbeit im Internet             
                                             &  & $\times$ \\\hline
  \end{tabular}
  
  \vspace{6em}
  \noindent\begin{tabular}{p{0.37\textwidth}p{0.56\textwidth}}
  \ort, \datum  & \rule{0.56\textwidth}{0.5pt}\\
                & \makebox[1cm]{\ } \autor
  \end{tabular}
  \cleardoublepage
  
   % Titelseite, Erklärungen, etc.

\begin{abstract} 

PDF-Dokumente besitzen viele Informationen, aus denen sich neue Daten generieren lassen können.
Doch das Extrahieren von solchen Daten ist heutzutage immer noch mit Hürden verbunden.
Dies gilt auch für die Dokumente, die in den Prozessen von Autohäusern und deren Kfz-Werkstätten verwendet 
und anschließend gelagert werden.
Um die Frage zu beantworten, inwiefern neue Daten aus dieser Art von Dokumenten verarbeitet werden können, 
wird im Rahmen dieser Arbeit ein System konzipiert und entwickelt, welches es ermöglichen soll, weiteres Wissen zu gestalten 
beziehungsweise zu generieren.
Dieses System will dabei zwei verschiedene Ansätze aus der künstlichen Intelligenz nutzen, um einen automatisierten Lösungsweg zu kreieren, und
eine grafische Schnittstelle für die Möglichkeit der menschlichen Nutzung anbieten.

\end{abstract}

\tableofcontents


\chapter{Einleitung}\label{chap:einf}

„Digitalisierung im Alltag voranbringen“ – Das war einer der Wahlslogans während der Bundestagswahl 2021. 
Gemeint war damit die Forderung nach einer zunehmende Digitalisierung im privaten Alltag vieler Bürger:innen, 
aber auch in der Wirtschaft machte sich wachsend der Wunsch nach mehr digitalen Alternativen breit (vgl. Bundesregierung 2021). 
Dieser Wunsch beinhaltete vor allem einen Wechsel von gängigen Papierformen verschiedenster Dokumente hin
zu denselben in digitaler Ausprägung. 

Genau jenes Bedürfnis nach Digitalisierung betrifft auch die vielen Arbeitnehmer:innen und Händler:innen in Autohäusern 
und deren Kfz-Werkstätten, die durch ihre beruflichen Tätigkeiten mit einer Vielzahl an Unterlagen, 
Dokumenten oder Belegen arbeiten müssen. Unter dieser Vielzahl fallen Dokumente wie Werkstatt-, Kauf- und Mietverträge 
sowie Rechnungen oder Diagnoseberichte. 

Jene Beschäftigte in einigen Autohäusern und deren Kfz-Werkstätten sind Kund:innen bei ilexius GmbH. 
Ilexius bietet Enterprise-Resource-Planning (ERP) Systeme an, mit Hilfe derer Arbeitsaufträge und Arbeitsschritte innerhalb des 
täglichen Arbeitsprozesses in Autohäusern und deren Werkstätten durch Digitalisierung 
vereinfacht werden. In Kooperation von ilexius GmbH werde ich daher die Problematik meiner Thesis, die ich im folgenden
noch genauer beschreiben werde, lösen und in die Tat umsetzen.

Durch eine Digitalisierung jener Dokumente eröffnen sich Vorteile wie bessere Zugänglichkeit, 
größere Langlebigkeit und vor allem eine angepasste und leichtere Nutzung, 
die zu einer höheren Effektivität innerhalb täglicher Arbeitsschritte führt.
Durch bisherige erste Schritte der Digitalisierung sind diese benötigten Dokumente 
bereits elektronisch aufgearbeitet und den Beschäftigten in Autohäusern und deren Kfz-Werkstätten zur Verfügung gestellt worden. 

Die Folge dessen ist, dass alle Dokumente standardisiert sind und dadurch die in den Dokumenten beinhalteten Informationen 
neu verarbeitet werden können. Eine Extraktion der Daten der einzelnen Dokumente ist jedoch nur eingeschränkt möglich, 
da die Struktur im gängigen Gebrauch im Dateiformat Portable Document Format (PDF) festgelegt ist. 

Jene Limitierung der Datenextraktion wirkt sich gleichermaßen auf die Dealer-Management-Systeme (DMS), 
mit solchen die Mitarbeiter:innen ihre Prozesse abwickeln, aus. Der daraus resultierende Arbeitsaufwand, 
welcher dabei entsteht, um die Daten wiederverwendbar zu konstruieren, ist derzeit enorm. 

\section{Zielsetzung}\label{sec:ziel}

Aus diesem Grund möchte ich innerhalb dieser Master-Thesis ein System entwickeln, 
das genau jene Problematik erleichtert und löst.
Innerhalb üblicher Methoden werden heutzutage die Daten zunächst über eine grafische Anzeige mithilfe 
einer bereits existierenden Benutzeranwendung via gängiges Kopieren und Einfügen entnommen. 
Dieser Schritt funktioniert zwar im Regelfall, ist jedoch oftmals stark fehleranfällig und kann lückenhaft sein. 

In so einem Fall muss das Einfügen und Kopieren manuell durchgeführt werden, 
was bei einer riesigen Menge an Dokumente zu einer monotonen Arbeit für die Händler der Autohäuser führt. 
Andernfalls können die Daten eigenhändig abgeschrieben werden – jener Vorgang benötigt jedoch viele Ressourcen.
Eher geeignet ist es, einen Datenkonverter für die PDF Dateien zu nutzen, um so die Dokumente beispielsweise in Bilder umzuwandeln, 
wodurch die Daten zugänglicher sind, aber immer noch verarbeitet werden müssen.
Die letztgenannte Möglichkeit wirkt allerdings eher wie eine Übergangslösung als eine professionelle endgültige Durchführung.

Aufgrund dieser bisherigen teils aufwendigen und mit Fehlern verbundenen Möglichkeiten möchte ich mich in meiner Thesis 
von diesen Optionen abwenden und das Data-Mining nutzen. 
Das Data-Mining soll eine Automatisierung unterstützen, mit der Daten aus dem PDF-Dokument extrahiert werden. 
Allein über das Data-Mining ist es möglich, Inhalte zu extrahieren und für andere Systeme bereit zu stellen. 

Der Unterschied zwischen meiner Verwendung des Data-Minings und zum gängigen Data-Mining 
liegt dabei in den Einschränkungen des Dateiformats: 
der Quelltext einer solchen Datei stellt erstens keine klare Hierarchie der Daten dar und zweitens 
besitzt es durch das Fehlen von Markierungen keine Informationen darüber, 
was die Daten an sich darstellen sollen~\cite{docsumoPDFScraperScrape2022}. 

Mit Hilfe dieser Technik des Data-Minings soll es ermöglicht werden, aktuelle Prozesse detaillierter zu steuern und zu überwachen. 
Des Weiteren können unter anderem Abgleiche von Rechnungen mit dem System durchgeführt werden 
und weitere Datenanalysen und Reporting kreiert werden.
Mit dem derzeitigen Stand der Datenextraktion können jedoch lediglich Agierende aus dem technischen Bereich arbeiten, 
da allein sie das dafür technisch notwendige Know-how besitzen. 
Die Technologien hierfür benutzen unterschiedliche Ansätze und müssen daher zunächst für diesen Fall ausgewählt werden. 

Damit jedoch auch für Agierende innerhalb des technischen Bereichs die Nutzung nicht zu abstrakt bleibt, 
soll durch die Entwicklung eines Systems der Zugriff greifbarer gestaltet werden, 
welches wiederum allein durch eine benutzerfreundliche Anwendung ermöglicht wird. 
Die Benutzer:innen sollen eine Möglichkeit bekommen und über eine Eingabe der Steuerung bestimmen können, 
welche Informationen extrahiert werden sollen.

Darüber hinaus muss eine Automatisierung vorhanden sein, um auch eine große Datenmenge verarbeiten zu können;
zudem sollen die Vorgänge eine niedrigere Fehlerquote aufzeigen. 
Die Prozesse müssen außerdem während des gesamten Ablaufs über eine Steuerung und Regelung kontrollierbar sein.
Solche Prozesse können zum Beispiel über Regeln festgelegt beziehungsweise gesteuert werden, 
die hingehen bei einer Erfüllung weitere Aktionen oder Regeln auslösen können.

Diese im vorherigen genannten Absatz beschriebene Eigenschaft kann durch eine symbolische künstlichen Intelligenz (KI) abgedeckt werden, 
welche eine vorgegebene Verarbeitung zulässt. 
Hiermit bezeichnet man einen altmodischen Ansatz der KI, bei der über das Festlegen von Symbolen 
menschliches Wissen in einer Logik gehalten wird und diese benutzt wird um weiteres Wissen sodann zu generieren~\cite{dicksonWhatSymbolicArtificial2019}.
Durch eine Parametrisierung einer solchen KI kann dann die Unschärfe der ausgewählten Daten festgelegt werden, 
damit eine Feineinstellung erfolgen kann. 
Das heißt, die Benutzer:innen eines solchen Systems soll sich beginnend von groben Definitionen für die Verarbeitung 
zu einer detaillierten und angepassteren Definition über Feedback des Systems hinarbeiten. 

Mit Hilfe dieser Annäherung an Definitionen und Regeln kann so für eine bessere Erfolgsquote gesorgt werden. 
Zudem kann der Ansatz des Machine Learnings (ML), mit jenem erfolgreiche Verarbeitungen einem System antrainiert werden, 
weitere Dokumente aus Daten extrahieren. 
Hierbei ist noch offen, welcher Typ des Machine Learnings sich für diesen Fall am optimalsten eignet.

Beide Ansätze – der der symbolischen KI und der des Machine Learnings – bieten als Option die Entwicklung 
einer grafischen interaktiven Entwicklungsumgebung (IDE) an. 
Die IDE soll das Festlegen von Definitionen und Regeln erlauben, mit welchen die Dokumente verarbeitet werden können. 
Außerdem soll diese Oberfläche verschiedene Funktionalitäten offerieren und auch Feedback sowohl bei einem problemlosen Lauf, 
als auch bei einem fehlerhaften Lauf, zurückgeben. 

Des Weiteren soll es das System es ermöglichen, Änderungen der Definitionen anzumerken und Unterschiede zu erstellen. 
Mit dieser Funktion sollen auch bisherige Ergebnisse angezeigt werden.
Die Festlegung von Definitionen und Regeln soll außerdem durch eine Ansicht der Dokumente unterstützt werden.
Insgemein wird das System die zwei Ansätze als Subsysteme aufteilen um die Umsetzung zu modularisieren 
und einen Vergleich zu ermöglichen.

\section{Ablauf}\label{sec:ablauf}

Um genau dieses optimale System für das beschriebene Szenario entwickeln zu können, werde ich in dieser Arbeit wie folgt vorgehen: 

Für eine vollständige Auseinandersetzung soll eine weitere Analyse stattfinden,
damit die vorhandenen Grundlagen für den Anwendungsfall, um den es in der Arbeit geht, evaluiert werden können. 
Das heißt, es werden mit Hilfe der Funktionalitäten der Bibliotheken Ergebnisse 
auf Basis der Datensätze erzeugt, um die Erfolgsquoten zu überprüfen. 
Diese werden dann verglichen und ausgewertet. Parallel dazu soll die Umgebung 
für die Entwicklung eingerichtet werden, mit der die Implementierung der Systeme stattfinden soll. 

Danach widme ich mich der Konzeption und der Entwicklung des Systems.
Diese beginnt mit der symbolischen KI als ein Subsystem des gesamtem Projekts, durch selbige ein direkterer Ansatz ermöglicht wird. 
Die Entwicklung hierbei wird im Wasserfall-Ansatz, einem Ansatz bei der phasenweise 
die Eigenschaften einer solchen KI umgesetzt werden, 
um eine Grundlage für die nächste Komponente zur Verfügung zu stellen.

Bei dieser Komponente handelt es sich um die interaktive Entwicklungsumgebung, die den Zugang für die Benutzer:innen zum System darstellt.
Deswegen folgt im nächsten Schritt die Konzeptionierung und Entwicklung der interaktiven Entwicklungsumgebung, 
welche sich während der Entwicklung der symbolischen KI parallelisieren lässt. 
Sobald das Gerüst der Benutzungsoberfläche fertiggestellt ist, möchte ich dies mit der symbolischen KI verbinden.

Als Gegenstück wird sich dann mit dem Machine Learning Ansatz und dem dazugehörigen Subsystem, 
bei der gleichermaßen eine Konzeptionierung und Entwicklung stattfindet, gewidmet.
Hier wird das bisherige Gesamtkonstrukt in einem iterativen Ansatz umgesetzt, 
sodass der Hauptteil so früh wie möglich verfügbar ist, 
damit das Training des ML-Systems durch die Dokumente auch zeitnah starten kann. 
Auf das Auswerten dieses Subsystems folgt eine Anpassung dessen. 
Die Komponenten der symbolischen KI und des Machine Learnings erlauben damit einen Vergleich 
zwischen zwei verschiedene Ansätze aus der Künstlichen Intelligenz.

Danach wird das Software Testing für die beiden Subsysteme eingeführt. 
Mit Unit Testing, Integration Testing und Functional Testing wird das erfolgreiche Ausführen gewisser Eigenschaften abgedeckt.
Dies dient wiederum als Grundlage dafür, die Continuous Integration (CI) für das System einzuführen, 
mit dieser eine sichere und konsistente Entwicklung nach dem Prototyping angeboten wird.  

Wenn hierauf die Systeme gegeben sind, kann das grafische Tool weiter angepasst werden, welches 
sich dann mit den Systemen verbinden soll, um infolgedessen die Ausführbarkeit der Funktionalität überprüfen zu können. 
Um ein gemeinsames System zu ermöglichen, welches dann gegebenenfalls weitere Testschritte bedarf,
werden anschließend die Komponenten zusammengeführt.  
Sobald jenes eine erfolgreiche Form annimmt, wird der Teil der Continuous Integration mit einbezogen, 
mit jener das System einen Zustand der Instandhaltung annehmen kann. 
Über Testing sollen so mögliche Fehlerquellen entdeckt werden, bevor diese überhaupt auftreten können.

Unter die Zielsetzung für meine beiden eben genannten Ansätze fallen die Kundendaten mit den Eigenschaften wie 
Name, Firma, Adresse und Kundennummer. Zu Eigenschaften der Fahrzeugdaten zählen Angaben der Fahrgestellnummer, 
Modell, Farbe, Motor, Erstzulassung, Letzter/nächster Service und 
TÜV sowie die Abrechnungsdaten wie die Jobs mit den dazugehörigen Arbeitspositionen, der Reparaturteile, den Preisen etc.

Dazu stehen als weitere Grundlage mehr als 10.000 annotierte unterschiedliche Dokumente mit genau den Daten, 
die extrahiert werden sollen, als Datenbank zur Verfügung. 
In Zukunft sollen für das Mengengerüst mehr als 100.000 Dokumente pro Jahr bearbeitet werden. 
Hierbei ist es aber auch möglich, aus den bereits bestehenden Daten einen Pool für den Lernprozess beziehungsweise 
für die Verifikation zu bilden.

Zum Schluss werde ich die Ergebnisse des gesamten Systems betrachten und einen Ausblick zu der Thematik geben.
Aus Datenschutzgründen werde ich in meiner Arbeit persönliche Informationen in den gezeigten Dokumenten anonymisieren und verpixeln.

\chapter{Hintergrund}\label{chap:hint}

Um die Problematik detaillierter darzustellen und um eine Übersicht zu ermöglichen, 
wird in diesem Kapitel der Rahmen der Problematik aus technischer Sicht genauer dargestellt.
Hierbei werden die möglichen Technologien abgewogen und evaluiert. 
Zunächst jedoch wird der Kontext ausführlicher erklärt.

\section{Kontext}\label{sec:kon}

Im Automobilbereich wird der Verkauf von Kraftfahrzeugen in zwei Segmente aufgeteilt: dem Sales-Bereich, 
bei dem es sich um den Verkauf 
von Neu- und Gebrauchtwagen handelt und dem After-Sales-Bereich, bei jenem eine Bindung zum Kunden über den Verkauf von 
Verschleiß- und Ersatzteilen geschaffen wird.~\cite{theuererWasMachtSales}
Diese Aufteilung der Segmente wird ebenfalls von den Autohäusern angewendet, zugleich werden für diese Aufteilung
jeweils gängige Abläufe als Prozesse benutzt.

Für den After-Sales-Bereich bietet ilexius GmbH zwei Arten von ERP-Systeme an. 
Die erste Art ist das von der Automobilmarke Jaguar Land Rover finanzierte \textit{vTab}, welches eine Plattform 
für elektronische Fahrzeugkontrollen 
und ein Backend für Jaguar Land Rover Mitarbeiter:innen, mit dem die Autohäuser bewertet werden können, anbietet.
Die zweite Art mit dem Namen \textit{ATT} ist eine Plattform. 
Mit derjenigen können Autohäuser ihre Arbeitsprozesse in Schrift, Sprache und Bild dokumentieren.
Die Dokumentation erfolgt mittels mobiler Endgeräte und werden an das ERP-System weitergeleitet, 
sodass die Dokumentation archiviert werden kann.

Die ERP-Systeme, die bereits im Kapitel~\ref{sec:ziel} beschrieben worden sind, unterstützen 
damit die Kunden:innen von ilexius GmbH 
in den Autohäusern und deren Werkstätten bei der Planung, der Steuerung sowie der Verwaltung 
von verschiedenen Aufgaben in ebenjenem After-Sales-Bereich.

Eine Art der Aufgaben sind Arbeitsaufträge, bei der es sich in den meisten Fällen um die Handhabung und Abfertigung 
eines Autos inklusive Kundendaten handelt. 
Der Auftrag besteht hierbei aus einer Liste an Arbeitsabläufen, die als Jobs bezeichnet werden. 
Die Jobs wiederum unterteilen sich in einzelne Arbeitsschritte, die Joblines genannt werden, 
welche die konkreten Verrichtungen der Arbeit darstellen und somit den gesamten Arbeitsauftrag 
in einzelne Teilschritte vervollständigen.

Sind diese Aufträge abgeschlossen, benutzen die Mitarbeiter:innen der Autohäuser ihre Dealer-Management-Systeme, 
um eine Rechnung zu erstellen und diese im ERP-System zu archivieren.
Das ERP bietet des Weiteren eine Schnittstelle für diese Dokumente, wie z.B die Rechnungen oder die Garantieaufträge an, 
die allein über Scanner die Dokumente übertragen. 
Dort werden die Dokumente verarbeitet und mit Hilfe eines OCR Scanners digitalisiert. 

Derzeit ist allerdings die Digitalisierung nicht in der Lage, einen Kontext beziehungsweise
eine Semantik für diese Dokumente zu erstellen.
Idealerweise kann dies genutzt werden, um aus den Dokumenten Arbeitsaufträge zu generieren und somit zu digitalisieren.
Demzufolge ergibt sich hier an der Stelle die Möglichkeit, einen neuen Ansatz zu testen, infolgedessen die Informationen 
aus diesen Dokumenten als Daten erhaltbar sind.

\section{Analyse}\label{sec:ana}

Das Extrahieren von Daten ist eine gängige Methode, um Informationen aus verschiedenen Systemen wie Datenbanken 
oder Software as a Service (SaaS) Plattformen zu erhalten. 
Durch diese Beschaffung können die Daten weiter verarbeitet werden, 
wodurch neue Lösungsmöglichkeiten für das eigene System bereitzustellen.
Die Arten des Extrahierens unterscheiden sich hierbei im Zeitverlauf und der Herangehensweise. 
So können sie extrahiert werden, sobald Änderungen der Daten beobachtet oder mitgeteilt worden sind. 
Durch diese Herangehensweise werden so die Daten schrittweise herangeschafft~\cite{stichdataWhatDataExtraction}.

Des Weiteren gibt es die Option, eine komplette Extraktion durchzuführen. 
Dieser Ansatz kann sich in vielen Fällen als nachteilig erweisen, da die Daten jedes mal 
bei Anpassungen der ursprünglichen Information neu verarbeitet werden müssen.
Aufgrund des Bezugs der Problematik im Rahmen der Thesis auf Dokumente, die bereits im ERP-System archiviert wurden, 
trifft dieser Fall hier nicht zu.

\subsection{Extraktionsbibliotheken}\label{sec:exwerk}

Für das Data Mining von PDF Dateien existieren bereits Lösungen, die von großen Unternehmen wie Amazon Web Services (AWS) 
oder Adobe angeboten werden. 
So nutzen Produkte wie Textractor~\cite{amazonIntelligenteExtraktionText} von AWS oder 
Adobes Extractor API~\cite{developersExtractTextPDF} künstliche Intelligenz, Machine Learning und Optical Character Recognition (OCR), 
um die Extraktion der Daten zu ermöglichen. 
Unternehmen wie ABBYY~\cite{abbyyPDFSoftwareOpen} haben sich bereits auf diesem Gebiet spezialisiert 
und sind erfolgreich im Umgang mit der Datenextraktion. 
Auch existieren bereits Open-Source Bibliotheken, die sich mit dem Entnehmen der Daten aus PDF Dokumenten auseinandersetzen.

Die meisten der Open-Source Bibliotheken lassen sich hinsichtlich der Aufteilung der Aufgaben gruppieren. 
So erlauben unter anderem PDFMiner~\cite{unixuserPDFMiner} oder Apache Tika~\cite{apacheApacheTikaApache}, 
welches vor der Extraktion der Texte und Metadaten noch die Klassifizierung und Identifizierung des Medientyps von Dateien 
anhand einer umfangreichen Hierarchie besitzt, Texte aus PDF Dokumenten zu gewinnen~\cite{mattmannTikaAction2012}. 
Funktionalitäten für das Extrahieren von Daten aus Tabellen bieten Projekte wie 
Tabula~\cite{tabulapdfTabulapdfTabulaTabula} oder Camelot~\cite{camelotCamelotPDFTable2022} an.
Diese Projekten decken ferner einen weiteren Aspekt durch die Nutzung der Funktionalitäten ab.

Bei den beiden zuletzt genannten Bibliotheken weisen sich niedrige Fehlerquoten~\cite{atlanhqComparisonOtherPDF} 
und die Nutzung von einer Fuzzy Logik auf. 
Camelot ermöglicht überdies ein Web Interface, mit dessen Hilfe eine simple Extraktion mittels einen Browser lokal stattfinden kann.
Weitere Bibliotheken, wie Textricator~\cite{measuresMeasuresforjusticeTextricatorTextricator} sind umfangreicher gestaltet 
und ergeben für die Anfragen spezifischer Inhalte geeignete Schnittstellen. 

Doch auch wenn dies mit textbasierten Dokumenten funktioniert und der Export flexibel ist, 
können zunächst weder nicht-textbasierte noch komplexere PDF Dokumente verarbeitet werden. 
Des Weiteren sind die Verarbeitungen nicht automatisierbar und die Nutzer:innengruppe 
wird durch eine fehlende grafische Komponente eingeschränkt.
Aus dieser Problematik resultierend, zeigt sich, dass community-angetriebene Projekte wie PdfMiner.six~\cite{pdfminerWelcomePdfminerSix}, 
das eine Abspaltung des bereits erwähnten PdfMiner ist, und PdfPlumber~\cite{singer-vinePdfplumber2022}, 
welches wiederum auf PdfMiner.six basiert, geeignete Alternativen darstellen. 

PdfPlumber stellt im Vergleich zu seinem Ausgangsprojekt überdies eine Tabellenextraktion bereit und bietet für beide Aspekte 
der Extraktion eine Schnittstelle an. 
Es stellt eine Kommandozeile, die Möglichkeit, visuelles Debugging zu nutzen, 
und Funktionalitäten wie das Entfernen von Duplikaten in Texten, zur Verfügung.

Das visuelle Debugging, welches in Abbildung~\ref{fig:vd} zu sehen ist, ermöglicht es unter anderem, 
die Texte des Dokuments hervorzuheben, was wiederum die Extraktion greifbarer darlegt 
und die Entwicklung mit den Funktionalitäten unterstützt.
Auch die Funktionalität des Entfernens von Duplikaten kann dann zum Einsatz kommen, wenn Dokumente bei ihrer Erstellung 
oder Verarbeitung versehentlich die selben Einträge nochmal im Quellcode des PDFs besitzen. 
In dieser Abbildung sind die persönlichen Daten des Auftrags anonymisiert und deshalb verpixelt.

\begin{figure}[htb]
    \centering
    \subfloat[ohne visuellem Debugging]{\label{fig:wovd}
    \frame{\includegraphics[width=.45\textwidth]{images/kfz3000_werkstattauftrag_pixelated.png}}}
    \subfloat[mit visuellem Debugging]{\label{fig:wvd}
    \frame{\includegraphics[width=.45\textwidth]{images/kfz3000_werkstattauftrag_vd_pixelated.png}}}    
    \caption{visuelles Debugging bei einem Dokument }
~\label{fig:vd}
\end{figure}

Für die Auswahl einer Extraktionsbibliothek ist auch die dazugehörige Programmiersprache für das System der Datenextraktion, 
in der die Bibliothek umgesetzt worden ist, relevant. 
Darunter fallen die bereits genannten Bibliotheken, welche sich in zwei Sprachen gruppieren lassen.
Projekte wie Tika, Tabula und Textricator sind mit Hilfe von Java implementiert, 
während die restlichen, aufgezählten Bibliotheken in Python geschrieben sind~\cite{10.5555/1593511}.
Für Tabula gilt hier hingegen die Ausnahme, dass es für das Projekt unterschiedliche Wrapper für weitere Programmiersprachen wie Python gibt.
Hieraus ist zu beobachten, dass die Nutzung von Python viele mögliche Extraktionswerkzeuge erlaubt, 
welches sich für die spätere Umsetzung als noch bedeutsam erweist.

Die Hürden für eine komplett automatische Extraktion bestehen jedoch nach wie vor.
So ist nämlich nicht ersichtlich, wie das zu erwartende Verhalten der Logik der Extraktion für den generellen Fall auszusehen hat.
Eine differenzierte Extraktion für weitere Eigenschaften eines Dokuments, wie zum Beispiel Zeilenumbrüche von Absätzen, 
die Seitennummern, die Kopf- und Fußzeilen, die Formatierung et cetera (etc.) ist daher nötig.
Die Frage, die sich dabei stellt, ist, welche Eigenschaften eines Dokuments konkret zusätzlich extrahiert werden sollen. 
Um eine Auswahl zu treffen, müssen die Arten der Dokumente, die in dem ERP-System auftreten, präziser durchleuchtet werden.

\subsection{Dokumentarten}\label{sec:dokart}

Bei den Dokumenten, die in das ERP-System eingescannt werden, handelt es sich in der Mehrheit um Garantie- und Werkstattaufträge und Rechnungen. 
Jedes Autohaus benutzt hierbei ein individuelles Muster beziehungsweise Format für die jeweilige Art des Dokuments.
Dabei ist jedoch der Aufbau dieser Dokumente zu Teilen indirekt durch die von den Autohaus genutzten Dealer-Management-Systeme vorgegeben.

Ebendaher passiert es, dass es Autohäuser gibt, welche das gleiche DMS verwenden 
und zudem darüber hinaus auch einen detailliert angepassten Dokumentenaufbau für ihre Aufträge benutzen.
Dem Umfang dieser Thesis geschuldet werde ich mich auf drei Dokumententypen von verschiedenen Autohäusern 
mit ihren jeweils genutzten DMS beschränken und anhand jener drei Typen mein Vorgehen exemplarisch demonstrieren.
Die Abbildungen der Dokumente in diesem Abschnitt zeigen in einigen Fällen persönliche Daten 
und wurden deshalb, wie in der Einleitung beschrieben, anonymisiert.

In Abbildung~\ref{fig:care_g} wird die erste Art eines Auftrags dargestellt.
Bei diesem digitalen Schriftstück handelt es sich zunächst von der Form her um eine Garantierechnung des Dealer-Management-Systems Care\@.
Das Dokument besitzt drei Bereiche, die relevante Informationen für einen Arbeitsauftrag enthalten.
In der oberen Hälfte befindet sich zunächst das Adressfeld, welches die Kundeninformationen im genormten Adressformat enthält. 
Unter dem Adressfeld befindet sich ein weiterer Abschnitt, in dem die Fahrzeugdaten festgehalten werden.

\begin{figure}[htb]
    \centering
    \frame{\includegraphics[width=.7\textwidth]{images/care_garantieauftrag_pixelated.png}}
    \caption{Exemplar eines Garantieauftrags}
~\label{fig:care_g}
\end{figure}

Hierbei ist zu beobachten, dass dieser Abschnitt sich mit einer Tabelle vergleichen lässt. 
Die einzelnen Einträge werden mit jeweils unterschiedlich großen Abständen verteilt dargestellt,  
gleichzeitig erfüllen sie eine Gesamthöhe pro Spalte beziehungsweise eine Gesamtbreite pro Höhe einer Tabelle.
Gleichwohl fällt auf, dass sich — im Gegensatz zur Anzahl der Zeileneinträge — die Anzahl der Spalteneinträge pro Zeile unterscheidet.

Zudem fehlen jegliche Einrahmungen beziehungsweise jegliche Trennlinien für Zeilen und Spalten.
Bei den Einträgen in diesem Abschnitt geht es um die Abrechnungsnummer, das Datum der Reparatur, 
den Fahrzeugmodellnamen, das Kfz-Kennzeichen, den zuletzt abgelesen Kilometerzählerstand, die Arbeitsauftragsnummer, der Fahrzeug-Identifikationsnummer beziehungsweise der
Vehicle Identification Number (VIN), die/die zuständige/n Autohausmitarbeiter:in und das Datum der Erstellung des Arbeitsauftrags.
Verpixelt und damit anonymisiert wurden die Felder des Kfz-Kennzeichens, der VIN und der/des zuständigen Mitarbeiter:in.

Im unteren Bereich wird eine Tabelle durch die Aufzählung der einzelnen Jobs und Joblines dargestellt.
Die Tabelle ist an der Kopfzeile zu erkennen, die die Jobs und Joblines in die Arbeitsnummer beziehungsweise in die Teilenummer, 
in die Bezeichnung, in die Menge des Arbeitszeitwertes, in den Preis und in den Betrag übersichtlich aufteilt.
Die Teilenummer beziehen sich hierbei auf die Identifikationsnummer der Reparaturteile.
Der Arbeitszeitwert ist dabei eine Einheit für die genaue Abrechnung der Arbeitszeiten, 
um eine Aufschlüsslung der Stunden zu ermöglichen.~\cite{BedeutungAWArbeitswert}
Die Menge wird hierbei als Reparaturzeit oder als Anzahl der Elemente angegeben.
Daneben existiert bei dieser Tabelle nur für die Kopfzeile eine untere Trennlinie.
Auch hier sind Probleme bezüglich (bzgl.) der Form der Tabelle zu erkennen.

Während die ersten vier Zeilen noch dezent erscheinen, besitzt die fünfte Zeile 
im Abschnitt mit der Bezeichnung einen längeren Text, sodass der restliche Text in weiteren Zeilen in der gleichen Spalte 
darunter geschrieben steht.
Die Abstände zwischen den Spalten insgesamt scheinen konsistenter zu sein, sind jedoch in der Abbildung~\ref{fig:care_g} 
in mehreren Zeileneinträgen Überläufe bei den Arbeitsnummern beziehungsweise den Teilenummern zu sehen.
In der letzte Zeile ist darüber hinaus zu erkennen, dass die Einträge zu Teilen auch verschoben dargestellt werden, 
was eine eindeutige Erkennung der Positionierung noch weiter erschwert.
Inhaltlich werden Jobs mit den dazugehörigen Joblines darunter angereiht dargestellt. 

Dies ist durch die Codierung der Arbeitsnummern beziehungsweise der Teilenummer zu erkennen, 
die Hierarchie ist durch das jeweilige Dealer-Management-System vorgegeben.
In der ersten Spalte können daher Codes von Jobs, Reparaturzeiten, Ersatzteilen, 
Fremdleistungen, Stempelzeiten und Kundenbeschwerden auftreten.

Für die Codierung ist keine standardisierte Spezifikation vorhanden, 
weshalb das für jedes Autohaus und seine Nutzung der DMS individuell definiert ist.
So sind die Codes der Jobs als eine fünfstellige Zeichenkette aus Ziffern bestimmt, die mit den Ziffern \textit{00} angefangen. 
Die Reparaturzeiten dagegen besitzen eine sechsstellige Zeichenkette aus Ziffern.
Unterhalb der Tabelle werden am Ende des Dokuments schließlich die gesamten Kosten aus den Betragseinträgen summiert dargestellt.
Diese Summe lässt sich wiederum in Kosten der Arbeit und der Teile aufteilen. 

\begin{figure}[htb]
    \centering
    \frame{\includegraphics[width=.7\textwidth]{images/formel1_rechnung_pixelated.png}}
    \caption{Exemplar einer Rechnung}
~\label{fig:formel1_r}
\end{figure}

In Abbildung~\ref{fig:formel1_r} wird eine Rechnung der Formel 1 DMS dargestellt.
Auch hier sind sämtliche personenbezogenen Daten verpixelt und anonymisiert worden.
Im Vergleich zur vorherigen Abbildung~\ref{fig:care_g} ist ein ähnlicher Aufbau zu sehen. 
Während das Adressfeld geläufig aussieht, sind jedoch die Fahrzeug- und Kundendaten diesmal 
in zwei unterschiedlichen Tabellen aufgeteilt.

Die erste Tabelle befindet sich im oberen Teil der rechten Seite des Dokuments, 
wo die einzelnen Kundeninformationen aufgelistet sind.
Die Tabelle für die Fahrzeuginformationen befindet sich im unteren Bereich der oberen Hälfte des Dokuments 
und ist aufgrund der vorhandenen Trennlinien eine klassische Tabelle.
Im letzten Teil der Formel 1 Rechnung befindet sich die Liste der Job und Joblines.

Hier fallen ähnliche Problematiken, wie die unterschiedlichen Abstände zwischen den Spalten der Einträge 
im Vergleich zu dem Care Garantieauftrag auf.
Auch existieren Überläufe der Einträge, allerdings treten diese nur in den Feldern der Bezeichnung auf.
Die Gesamtkosten werden dann am Ende des Dokuments in den Teilsummen der Beträge bekannt gegeben.

\begin{figure}[htb]
    \centering
    \frame{\includegraphics[width=.7\textwidth]{images/kfz3000_werkstattauftrag_pixelated.png}}
    \caption{Exemplar eines Werkstattauftrags}
~\label{fig:kfz3000_w}
\end{figure}


Bei der dritten und damit letzten Dokumentart (Abbildung \ref{fig:kfz3000_w}) handelt
es sich um einen Werkstattauftrag von dem Dealer-Management-System KFZ3000.
Auch dieses hier abgebildete Dokument wurde aus Datenschutzgründen anonymisiert. 
Dieser hier gezeigte Auftrag lässt sich in Bezug auf die Aufteilung der einzelnen Segmente mit dem Care-Warranty-Dokument vergleichen. 
Während auch das Adressfeld erneut normiert abgebildet ist, werden Kunden- und Fahrzeugdaten 
über Schlüssel-Wert Anordnungen in einem größeren Feld dargestellt.

Die untere Hälfte des Werkstattauftrags besitzt gleicherweise wie die bereits anderen genannten Dokumente 
eine tabellenähnliche Auflistung der Job und Joblines. 
Ebenfalls sind hier Überläufe und unterschiedliche Abstände in dem Bereich zu erkennen.
Hinzu kommt, dass, im Gegensatz zu den anderen Dokumentarten, 
zwischen den Zeilen ein Art Summenstrich hinzugefügt wird, der die Anzahl der Jobs anzeigen soll.
Außerdem wird im unteren Bereich des Dokuments lediglich die gesamte Betragssumme gezeigt.

Insgesamt lässt sich feststellen, dass alle hier vorgestellten Dokumentarten und ihre Hürden
durch eine gemeinsame Problematik miteinander verbunden ist.
Die Darstellung der Kunden- und Fahrzeuginformationen wird entweder in einer Tabellenform 
gemeinsam (siehe Abbildung~\ref{fig:care_g}) oder in zwei verschiedenen Tabellen (siehe Abbildung~\ref{fig:formel1_r}) abgebildet.
Diese Tabellenform besitzt außerdem selten Trennlinien, welche geeignet sind, um eine derartige Tabelle als solche definieren zu können.
Für die Liste der Job und Joblines tritt das gleiche Problem auf.

Die Auflistung der Job und Joblines als Tabelle wird allein durch ihre Anordnung der Einträge 
und deren Abstände dazwischen als solche erkenntlich, 
jedoch bricht sie in den prototypischen Dokumenten aber mindestens (mind.) 
einmal die Form durch Überläufe oder Verschiebungen der Texte.
Ein weiterer relevanter Aspekt, welcher übersehen werden kann, ist, dass die Aufträge mehr als eine Seite haben können.

Dies tritt für Dokumente auf, welche eine längere Auflistung der Job und Joblines besitzen 
und somit auf weiteren Seiten dargestellt werden muss.
Für die Extraktion der Liste muss daher auch überprüft werden, 
ob sich diese über weitere Seiten erstreckt um sie zusammenzuführen und zu verarbeiten.
Es ist festzustellen, dass verschiedene Tabellenstrukturen für unterschiedlichen Dokumentarten existieren.
Dieses Angelegenheit besteht, da bisher für Tabellen keine universelle Standardisierung definiert ist.

Damit diese Formen durch eine Logik des Extrahierens abgedeckt werden, besteht die Möglichkeit der Fallunterscheidung.
Bei dieser Unterscheidung sollen die Dokumente anhand ihrer Arten konzipiert und definiert werden, 
sodass die Entnahme detaillierter stattfinden kann.
Für andere Bereiche eines Dokuments wie dem Adressfeld werden gewöhnliche Funktionalitäten der Extraktion genutzt und zur Verfügung gestellt.
Diese können dann von allen weiteren Dokumentarten geteilt und genutzt werden, 
sodass eine Verallgemeinerung verschiedener Dokumententypen ermöglicht wird.


\subsection{Frameworks}\label{sec:frameworks}

Damit das System für den Prozess der Extraktion als solches sich zeitnah im Rahmen einer Thesis umsetzen lässt, 
können Frameworks dem ganzen Zeitfenster entgegenkommen.
Im Zuge dessen werden für die Teilsysteme verschiedene Technologien betrachtet und in Erwägung gezogen.
So wird zunächst die symbolische künstliche Intelligenz dem ersten Teilsystem zugeordnet und in seiner Art unterschieden. 
Zudem müssen die Realisierungsmöglichkeiten der symbolischen KI betrachtet 
und aus den verschiedenen Arten eines wissensbasierten Systems muss ein Modell ausgewählt werden.

Ein solches wissensbasiertes System ist hierbei der Oberbegriff für Programme, die mit Hilfe einer Wissensbasis Mechanismen der Schlussfolgerung nutzen, 
um Lösungen für Probleme im dafür gewählten Anwendungsbereich zu finden~\cite{embreySkillRuleKnowledge}. 
Es existieren verschiedene Modelle eines wissensbasierten Systems, die abhängig vom Anwendungsfall passend sind.
Folgendermaßen gibt es den Ansatz des regelbasierten Systems beziehungsweise des Produktionssystems, bei dem die Schlussfolgerung über Regeln und Fakten, 
die von Benutzer:innen definiert werden, konstruiert wird.

Bekannt für die Erstellung eines solchen Systems ist das in der Programmiersprache C geschriebene Werkzeug \textit{CLIPS}\cite{clipsCLIPSToolBuilding}. 
Dieses Werkzeug verfügt über eine eigene Syntax, welche von der Syntax  
des von der Inference Corporation entwickelten Expertensystems \textit{ART} für die Regeln 
und Fakten abgeleitet wurde und damit zahlreiche Gemeinsamkeiten beziehungsweise Ähnlichkeiten aufweisen.~\cite{haleyaiHaleyARTSyntax2008}. 
Die Umsetzung wird ebenso mit dem Wrapper \textit{clipspy} für die Programmiersprache Python zur Nutzung bereitgestellt.

Eine Alternative zu \textit{CLIPS}, die zeitlich danach entwickelt wurde, ist das Expertensystem \textit{NEXPERT OBJECT}, 
das sich bis auf die Datenbankintegration nur marginal in der Nutzung der Regel- und Faktendefinitionen unterscheidet~\cite{https://doi.org/10.1111/j.1468-0394.1990.tb00164.x}
und obendrein keine aktive Entwicklung und Wartung existiert. 
Andere moderne Umsetzungen konzentrieren sich im Vergleich eher auf die als Regelinterpreter verstandene Inferenzmaschine.
Projekte wie \textit{Pyke}~\cite{frederiksenWelcomePyke}, \textit{Durable Rules}~\cite{ruizDurableRules2022} oder \textit{Experta}~\cite{perezNilp0interExperta2022}, 
dem Nachfolger von einem Expertensystem Framework namens \textit{pyKnow}, erlauben es, eine Regelmenge zu definieren 
und die einzelnen Regeln mit selbstdefinierten Funktionen zu koppeln.

Für die Entwicklungsumgebung, die den Zugang zum System darstellen soll, 
eignen sich sowohl Nutzungsschnittstellen beziehungsweise User Interface (UI) Bibliotheken für den Desktop als auch für den Webbereich.
Die Priorität liegt hierbei bei der Erstellung der einzelnen Komponenten der graphischen Benutzungsoberfläche beziehungsweise der Graphical User Interface (GUI). 
Folglich können Projekte wie \textit{Qt}~\cite{qtQtCrossplatformSoftware} und und das in C++ geschriebene \textit{ImGui}~\cite{omarDearImGui2022} Bibliotheken bereitstellen, 
die auf einer nativen Ebene funktionieren und demgemäß in verschiedenen Anwendungsfällen genutzt werden.
Die letztere genannte Bibliothek benutzt intern einen Zustandsautomaten, um die Render-Pipeline darzustellen  
(diese erfordert mehr Erfahrung mit der Grafikprogrammierung), dass jenes verwendet werden kann~\cite{furhtCreatingUserInterface2018}.

Dem gegenüber stehen abstraktere im Front-End Bereich verwendete UI-Lösungen wie das Benutzen eines Web Frameworks.
Bekannte Bibliotheken wie \textit{Vue} und \textit{React} ermöglichen eine schnelle und gängige Umsetzung einer Oberfläche, 
die auch die Zwecke einer Entwicklungsumgebung erfüllen können. 
Weitere Kandidaten wie \textit{Svelte}~\cite{svelteSvelteCyberneticallyEnhanced} versuchen durch das Nutzen eines Compiler, 
die Arbeit von dem Browser auf die Build Zeit umzulagern.

Für das Teilsystem, welches den Ansatz des maschinellen Lernens nutzen soll, 
sind gleicherweise hier ML-Frameworks eine geeignete Lösung.
Kandidaten sind die zurzeit konventionellen Frameworks wie \textit{TensorFlow}~\cite{tensorflowTensorFlow}, 
\textit{pyTorch}~\cite{pytorchPyTorch}
und \textit{spaCy}~\cite{explosionSpaCyIndustrialstrengthNatural}.
Jene bieten für die Anwendungsgebiete unter anderem in der Bilderkennung, 
der Computer Vision und des Neuro-Linguistische Programmieren (NLP) eine automatisierte Lösung an.
Weitere neue Projekte wie \textit{Fonduer}~\cite{wu2018fonduer} fokussieren sich mit Hilfe von schwachem überwachten Lernen (weak supervised learning) 
auf die Extraktion von umfangreichen formatierten Daten, worunter auch PDF Dokumente fallen.
Dieser Lernansatz findet auch bei nicht vollständig beschrifteten Trainingsdaten einen Anwendungszweck~\cite{10.1093/nsr/nwx106}, 
weswegen sich das für einfache Dokumente als praktisch erweisen kann.


\chapter{Konzeption und Umsetzung}\label{chap:konundum} %KAPITEL

Für den Sprung zur Umsetzung bedarf es als nächstes die Konzeption des technischen Rahmens. 
Zunächst wird der Aufbau des Projekts vorgestellt, wie die beschriebene Problematik gelöst werden soll.
Danach wird der Verlauf der Entwicklung über die Aufteilung des Systems in seine Komponenten dargestellt. 
Jene beschreiben im Detail ihre Eigenschaften und Zusammenhänge.

\section{Aufbau}\label{sec:auf}

Für die Struktur des Systems ist vorerst die Untergliederung des Projekts relevant. 
Sie kann je nach einem ausgewählten Architekturmuster unterschiedlich gestaltet werden. 
Da die Teilsysteme wiederum als eigene Systeme dargestellt werden können und für die einzelnen Anwendungsfälle kommunizieren beziehungsweise 
interagieren müssen, ist hier der Ansatz der verteilten Systeme geeignet.
Die Kategorie der verteilten Systeme ist in weitere unterschiedliche Architekturmuster gegliedert.
Eines dieser Muster ist die Client-Server Architektur, die es erlaubt, über ein Netzwerk die Kommunikation bereitzustellen 
und die Teilsysteme und ihre Aufgaben zu fraktionieren. 

Dahin gehend gibt es drei Komponenten mit unterschiedlichen Aufgaben, die in den eben beschriebenden Teilsystemen abgebildet werden: 
Die erste Komponente befasst sich mit der Extraktion über ein regelbasierte Logik, 
mit jener die Daten von mehreren Dokumenten eines Dokumententyps automatisiert erhalten werden.
Die Zweite setzt die Entwicklungsumgebung um und ermöglicht dadurch auch jegliche Erstellung 
und Bearbeitung von Regeln über Textdateien.  
Die dritte und letzte Komponente setzt für die Extraktion der Dokumente den Lösungsansatz des Machine Learnings um.

Innerhalb des Architekturmusters fungieren die erste und dritte Komponente als Server, 
die die Logik der Extraktion ausführen sollen. 
Die Entwicklungsumgebung als zweite Komponente ist im Gegensatz dazu als Client zu betrachten, 
der den Benutzer:innen Änderungen in Bezug auf Regeldefinitionen erlaubt. 
Damit dieser Netzwerkansatz für die Architektur umgesetzt werden kann, müssen die Komponenten Schnittstellen 
auf der Netzwerkebene bereitstellen.

Für die jeweiligen Komponenten, die sich mit der Extraktionslogik beschäftigen, 
dient hier der Lösungsansatz der Benutzung eines Webservers, welcher die Anfragen verarbeitet 
und die verschiedenen Zustände als Antwort den Anfragen zurückgibt.
Bei der Komponente der Entwicklungsumgebung eignet sich der Ansatz einer Webanwendung, 
mit welcher weitere Technologien aus dem Webbereich genutzt werden können.
Somit wird jegliche Bearbeitung der Definitionen durch die IDE als Anfrage verschickt 
und die daraus folgenden Antworten als Ergebnisse präsentiert.

Für die praktische Umsetzung oder für die Entwicklung wird der Lösungsentwurf 
als ein Softwareprojekt in Form eines Monoliths konstruiert, welches das System als ganzes darstellt.
Das Projekt trägt den Namen \textit{Smart Document Extractor} und wird mit dem gängigen Versionsverwaltungssystem Git aufgesetzt.
Die darin gegliederten Komponenten werden \textit{Rule System}, \textit{IDE} und \textit{ML} genannt.

Demgemäß wird das Projekt als lokales Repository und als Remote Repository auf dem GitLab Server des Unternehmens festgehalten.
Im Wurzelverzeichnis wird zudem mit der Container Virtualisierung von Docker 
eine betriebssystem-agnostische Umgebung für die Teilsysteme definiert.
Dies ist nicht nur der Softwareentwicklung, sondern auch der Continuous Integration und deren Testschritte dienlich
und ermöglicht einen Ausbau der Skalierung der Teilsysteme.

\begin{figure}[htb]
    \centering
    \frame{\includegraphics[width=.7\textwidth]{images/projekt_aufbau.png}}
    \caption{Struktur des Softwareprojekts}
~\label{fig:projekt_aufbau}
\end{figure}

Die einzelnen Teilsysteme werden mit Hilfe einer Docker Compose Datei anschließend orchestriert, 
sodass die Container gemeinsam gestartet werden und ihre Kommunikation über ein gemeinsames Netzwerk geführt werden kann.
Die zusätzliche Konfiguration von Umgebungsvariablen in dem Projekt erlaubt es außerdem, 
sensible Daten, wie die Anmeldeinformationen der Zugänge der Teilsysteme 
oder weitere Dienste für den Arbeitsbereich dynamisch anzupassen.
Die Konfiguration wird hierbei von dem Versionsverwaltungssystem ignoriert.

Abbildung~\ref{fig:projekt_aufbau} zeigt eine Übersicht des Projekts 
und seine Eigenschaften, auf die im weiteren Verlauf dieser Arbeit detaillierter eingegangen wird. 
Dort sind neben den beschriebenen Konfigurationsdateien die Teilsysteme und deren Module als Komponenten abgebildet.


\section{Symbolische KI}\label{sec:symki}

Für das Teilsystem der symbolischen künstlichen Intelligenz habe ich den Ansatz beziehungsweise
die Nutzung eines Wissensbasierten Systems ausgewählt. 
Dieser Ansatz ermöglicht die Steuerung der Extraktion über definierte Fakten und Regeln.
Wie in Kapitel~\ref{sec:frameworks} beschrieben, gibt es viele Frameworks, die ein solches System bereitstellen können.
Doch zeigt sich auch bei jenen, dass die Nutzung eines dieser Frameworks nicht für den Fall geeignet ist, 
da mehr Funktionalitäten als benötigt abgedeckt werden.

Ein regelbasiertes System zeichnet sich durch eine Anzahl von Regeln (sogenannte Produktionsregeln), Fakten, 
welche sich während des Inferenzsvorgangs verändern, und durch den Regelinterpreter aus.
Der Interpreter kann mit Hilfe dieser Menge die Schlussfolgerungen in Form einer 
Vorwärtsverkettung oder Rückwärtsverkettung durchführen~\cite{5709916}. 

Für den Kontext der Dokumentenextraktion bedarf es allein die Ausführung von Extraktionen und die Generierung neuer Fakten, 
sobald die benutzerdefinierten Regeln mit den bereits vorhandenen Fakten erfüllt wurden, als Funktionalitäten.

Die in dieser Arbeit vorgestellten Bibliotheken gehen mit einem größeren Aufwand der Implementierung einher, 
da jene Bibliotheken weitere Softwareabhängigkeiten zur Folge haben.
Daher wird hier der Lösungsweg über ein eigens implementiertes Regelsystem gewählt.
Der Regelinterpreter wird speziell für den Zweck der Datenextraktion entwickelt 
und passt sich damit an die Nutzung des gesamten Projekts an.
Für die Implementierung dieses Teilsystems wird sich zudem für die Benutzung der Programmiersprache Python entschieden.

\subsection{Extraktionsvorgang}\label{sec:exvor}

Bevor auf die Regelmaschine genauer eingegangen werden kann, muss zunächst die konkrete Extraktion beleuchtet werden.
Im Abschnitt~\ref{sec:dokart} wurde gezeigt, dass die Hauptproblematik der Dokumente in der Extraktion der Tabellen liegt.
Diesbezüglich definieren wir Tabellen, die für jede Zeile und Spalte Trennlinien besitzen, als konventionelle Tabellen.
Ein Beispiel hierfür ist die Tabelle, die in Abbildung~\ref{fig:formel1_r} die Fahrzeugdaten beinhaltet.
Tabellen, die keine oder nur zu Teilen Trennlinien für die Zeilen und Spalten nutzen, werden als unkonventionelle Tabellen definiert.
Diese Art der Tabelle tritt in den drei vorgestellten Dokumentarten in Form der Auflistung von den Jobs und Joblines auf.
Andere Tabellen, die eine beliebige Form und somit die restlichen Tabellenarten darstellen, werden als benutzerdefinierte Tabelle bestimmt.

Textfelder können im Vergleich zu den Tabellen leichter extrahiert werden, 
weswegen die Funktionalität für alle Dokumentarten beziehungsweise Dokumententypen benutzt werden kann.
Für eine Verallgemeinerung der Extraktionsfunktionalitäten wird daher eine gemeinsame Klasse namens \textit{DocumentExtractionType} definiert. 
Diese gemeinsame Klasse fasst als Obermenge die Methoden, die für alle Dokumentenarten als Unterklasse genutzt werden können, 
zusammen. Sie bietet die Möglichkeit an durch weitere Klassendefinitionen, 
die den Mechanismus der Vererbung nutzen, Abbildungen von unterschiedlichen Dokumententypen zu kreieren.
So werden daraus folgend die Dokumentarten, die im Kapitel~\ref{sec:dokart} vorgestellt wurden, 
durch die Klassen \textit{CareWarranty}, \textit{Formel1Invoice} und \textit{KFZ300Invoice} chronologisch
dargestellt.

Die gemeinsamen Funktionalitäten, die in der Oberklasse definiert sind, bestehen aus der Textextraktion von Textfeldern, 
der Tabellenextraktion von konventionellen Tabellen, der Textextraktion von unkonventionellen Tabellen 
und der Textextraktion von benutzerdefinierten Tabellen.
Damit diese Funktionalitäten ausgeführt werden können, müssen Eigenschaften der Extraktionsbereiche als Parameter angegeben werden.
Für alle beschriebenen Ziele existiert mindestens die Positionierung als Eigenschaft.
Sie kann als Bounding Box (bbox) mit den X und Y Koordinaten für die jeweiligen Extraktionsziele aufgeführt werden.

Eine weitere Eigenschaft, die sich aus der Positionierung schließen lässt, ist die Dokumentenseite. 
Bei den selektierten Dokumenten treten jeweils mehrere Seiten auf, wenn die Liste der Jobs und Joblines so viele Einträge besitzt, 
dass diese nicht auf eine Seite passen. 
Aus diesem Grund ist die Seitennummer,die den Funktionen neben den Bounding Boxen mitgegeben werden, relevant für die Beschreibung des Zielbereiches. 
Diese Seitennummern können den Funktionen neben den Bounding Boxen mitgegeben werden.
Daraus folgend werden diese zwei Eigenschaften — die der Positionierung und die der Seitennummer — als notwendige Bedingung für jegliche Extraktionsbereiche behandelt und 
damit für die Logik der Methoden durch die Parametereingabe als gegeben gesehen. 

Für die Umsetzung der einzelnen Funktionen wird die vorgestellte Bibliothek PdfPlumber benutzt, 
um für die jeweiligen Extraktionsziele mit verschiedenen Methoden die Daten zu erhalten. 
Die Methoden nutzen dann die Angaben der Seitennummer und der Bounding Box, 
sodass der Bereich der Extraktion zunächst eingeschränkt werden kann.
Mit dem ausgewählten Bereich werden dann die Funktionalitäten für die Text- oder Tabellenextraktion aufgerufen.

Bei den prototypischen Dokumenten fällt vorerst die hinderliche Tabellenextraktion von unkonventionellen Tabellen 
und indirekt von benutzerdefinierten Tabellen auf:
Die Bibliotheken können diese Tabellen mit den dafür konzipierten Funktionen nicht erkennen 
beziehungsweise von diesen keine strukturellen Daten, die die Tabellen abbilden sollen, erhalten.
Um dies zu umgehen, werden als Lösungswege die unkonventionellen und die benutzerdefinierten Tabellen 
zunächst als reine Texte extrahiert und danach werden die Einträge der jeweiligen Felder mit Hilfe von regulären Ausdrücken gefiltert.

Auch wenn der Bereich durch die Bounding Box umrahmt ist, kann der Bereich nicht relevante Texte besitzen, 
die die Verarbeitung erschweren.
Demzufolge werden weitere Eigenschaften wie die Angabe eines Start- und Endankers benötigt, um lediglich die Zeilen aus dem Rohtext zu erhalten.
Da bei den elektronisch verarbeiteten Dokumente die Tabellen als Texte mit Zeilenumbrüchen verfasst sind, 
können die Zeilenumbrüche genutzt werden, sodass aus den Zeilen eine Liste generiert wird.
Die Liste wiederum wird mit regulären Ausdrücken weiter verarbeitet. 
In Folge dieser Verarbeitung erhalten die Daten durch Zuweisungen eine strukturierte Form 
und stellen folglich die Semantik der Tabellen dar.

Da auch Dokumente mit mehreren Seiten existieren, bedarf es neben den Start- und Endanker eine Menge von Zwischenankern, 
die als Brücke für die längeren Listen dienen.
Für die unkonventionellen und benutzerdefinierten Tabellen werden daher neben der Positionierung und der Seitennummern 
die Start- und Endanker als die Voraussetzung für jene Extraktionsfunktionalitäten benutzt. 
Optional bleiben die Angabe der Zwischenanker.

Die Klassen der einzelnen Dokumentarten sind für weitere Funktionalitäten, die spezifischere Zwecke 
für den jeweiligen Dokumenttyp erfüllen und eine weitere Semantik für die extrahierten Texte und Tabellen bilden.
Hier sind weitere Eigenschaften wie die Reihenfolge der Jobs für die Zuweisung der Werte relevant.
Diese Methoden erhalten mit Hilfe von regulären Ausdrücken die einzelnen Werte für die gewünschten Felder 
und weisen jene mit ähnlichen Algorithmen den Feldern zu. 
So werden die Daten je nach Dokumentenart detaillierter erfasst und als komplexe Objekte verarbeitet.

\subsection{Webserver}\label{sec:webserv}

In Kapitel~\ref{sec:auf} wurde die Struktur des gesamten Systems definiert und die Client-Server Architektur festgelegt. 
Damit die Komponente der symbolischen künstlichen Intelligenz die Kommunikation über Netzwerke erlaubt, 
bedarf es die Implementierung eines Webservers.
Insgesamt gibt es viele Ansätze, wie ein solcher Webserver aufgesetzt werden kann.
Da aber die benötigten Funktionalitäten für die prototypische Entwicklung sich auf die Verarbeitung 
von gewöhnliche Hypertext Transfer Protocol (HTTP) Anfragen und Antworten beschränken, 
reicht die Nutzung eines Web Server Gateway Interface (WSGI) Servers aus.

Für diesen Zweck eignet sich das Mikro-Framework Flask, 
welches eine zeitnahe Umsetzung und Integration in das System ermöglicht~\cite{grinberg2018flask}.
Über die Umgebungsvariablen erhält daraufhin die Flask Instanz ihre Daten bezüglich der Konfiguration für das Hosting.
Damit diese Webserver Anwendung auch das Testing und die Continuous Integration unterstützt, 
wird hier das von der Dokumentation vorgegebene Muster für die Application Factory angewendet.

Bei Verfolgung dieses eben genannten Ansatzes wird eine Factory Funktion geschrieben, die neben der Kreierung einer Instanz  
die weitere Konfigurationen je nach Anwendungsfall erlaubt.
Für den hauptsächlichen Verwendungszweck wird hier zunächst die Regelmaschine instanziiert.
Anschließend werden die Route Funktionalitäten deklariert~\cite{flaskAPIFlaskDocumentation}.
Die Routes wiederum filtern die Anfragen nach den relevanten Daten, 
um diese an die Regelmaschinen-Instanz und ihren Funktionen weiterzugeben.

\subsection{Syntax und Grammatik der Fakten und Regeln}\label{sec:sug}

Die relevanten Daten der Dokumente können nun durch den Extraktionsvorgang erhalten werden.
Damit die Steuerung beziehungsweise die Anfrage für die Datenextraktion den Benutzer:innen zugänglich gemacht wird, 
muss eine geeignete Syntax für die Regel- und Faktendefinition festgelegt werden.

Die Fakten werden deshalb zunächst als eine Menge der zu extrahierenden Objekte definiert.
So wird eine Obermenge abgebildet, die als Hierarchie eine Teilmenge von Textfeldern, eine Teilmenge von konventionellen Tabellen, 
eine Teilmenge von unkonventionellen Tabellen und eine Teilmenge von benutzerdefinierten Tabellen besitzt. 
Die Teilmengen inkludieren jeweils auch die leere Menge und besitzen als Elemente die Extraktionsziele, 
welche sich an der Namenskonvention vom Python Style Guide orientieren, und somit einen beliebigen Namen tragen können~\cite{pep8}.
Für jedes Extraktionsobjekt gibt es benötigte und optionale Eigenschaften, denen Werte zugewiesen werden.
Ein Fakt ist daher eine Wertezuweisung einer Eigenschaftes eines Extraktionsziel.

Die Regeln werden neben den Fakten als weitere Obermenge definiert. 
Diese Menge lässt sich als eine Liste von Regeln darstellen, welche nach den selben Namenskonventionen der Fakten benannt werden.
Als Element in der Regelmenge besitzt jede Regel einen Regelausdruck, eine Regelaktion und eine Liste von Regelparametern, 
jene abhängig von der Regelaktion angegeben werden. 
Die Namen der Regelaktionen müssen dabei mit den Namen der gegebenen Funktionalitäten übereinstimmen, 
damit diese später von der Regelmaschine erkannt und mit den angegebenen Regelparametern ausgeführt werden können.
Die Regelausdrücke werden als ein String angegeben und müssen zunächst syntaktisch verarbeitet werden.
Hierfür existieren verschiedene Syntaxansätze.

Regelbasierte Systeme wie \textit{CLIPS} nutzen, wie bereits beschrieben, eine angepasste Version der Syntax aus der Programmiersprache Lisp.
Diese Syntax bildet mit Hilfe von Schlüsselwörtern die Ausdrücke für Regeln und Fakten.
Python ermöglicht es Texte mittels der integrierten \(eval()\) Funktion zu evaluieren und, falls es sich um eine Python Aussage handelt, 
diese auch auszuführen.
Die Funktion \( literal\_eval()\) aus dem Python-Modul \textit{Abstract Syntax Tree} (ast) besitzt 
die selbe Funktionalität mit einer eingeschränkten Untermenge für die Python Ausdrücke.
Hierbei werden bei beiden Funktionalitäten die Texteingaben mit Hilfe des Parsings in abstrakte Syntaxbäume abgebildet, 
die die daraus generierten Terminale und Operatoren beinhalten.
Bibliotheken, wie die von Google entwickelte Software Common Expression Language (CEL) erlauben 
durch das Auswerten von Ausdrücken die Bildung einer Semantik~\cite{googleCommonExpressionLanguage}.
Hierzu wird eine Grammatik benutzt, dergleichen die Ausdrücke mit Hilfe von Regeln in ihre Typen und Namen als Bestandteile aufsplittet.

\begin{listing}[htbp]
    % \lstset{basicstyle=\small}

    \begin{lstlisting}
        rule_grammar = r"""
        ?start: disjunction
        
        ?disjunction: conjunction
                      | disjunction "or" conjunction    -> or_
        
        ?conjunction: expression
                      | conjunction "and" expression    -> and_
        
        ?expression: expression operator expression
                     | value
        
        ?operator: comp_operator        
                   | ident_operator     -> ident_operator
                   | memb_operator      -> memb_operator
        
        ?comp_operator: "<"         -> lt
                        |">"        -> gt
                        |"=="       -> eq
                        |">="       -> ge
                        |"<="       -> le
                        |"!="       -> ne
        
        ?ident_operator: "is"           -> is_
                         |"is" "not"    -> is_not
        
        ?memb_operator: "in"            -> in_
                        |"not" "in"     -> not_in
        
        OBJECT_NAME: CNAME
        PROPERTY_NAME: OBJECT_NAME ("." OBJECT_NAME|NUMBER)* 
        CHAR: /[\x27].[\x27]/
        
        ?value: PROPERTY_NAME        -> property_name
                | "True"            -> true
                | "False"           -> false
                | "null"             -> null
                | NUMBER             -> number
                | CHAR               -> char
                | ESCAPED_STRING     -> string
        
        %import common.CNAME
        %import common.NUMBER
        %import common.ESCAPED_STRING
        %import common.WS
        %ignore WS
        """
    \end{lstlisting}
    \caption{kontextfreie Grammatik für die Regelevaluation}
    \label{code:grammar}
    \end{listing}

Das für dieses Projekt entwickelte regelbasierte System für die Komponente der symbolischen KI 
bedarf Regeln in Form von Booleschen Ausdrücken.
Damit das regelbasierte System die Ausdrücke für die Regeln und Fakten gleicherweise auswerten kann, 
bedarf es eine für den Anwendungsfall angepasste Definition einer Grammatik.
Zu diesem Zweck nutze ich die Parsing Bibliothek Lark. 
Jene erlaubt die Erstellung einer kontextfreien Grammatik in der Erweiterten-Backus-Naur-Form (EBNF). 
Basierend auf dieser Bibliothek werden unter der Nutzung eines LALR-Parsers 
abstrakte Syntaxbäume aus Texteingaben generiert~\cite{lark-parserLarkparserLarkLark}.
Im Listing~\ref{code:grammar} wird die implementierte Grammatik für das regelbasierte System als Rohstring dargestellt.
Dieser Rohstring wird dann dem Parser übermittelt, sodass jeglicher Regel- und Faktenausdruck durch einen abstrakten Syntaxbaum 
der Logik für die Evaluation dieser Regel bereitgestellt wird.
Die booleschen Ausdrücke werden in der Syntax von Python genutzt, infolgedessen die Evaluation direkt stattfinden kann.

Der Aufbau der umgesetzten  Grammatik basiert auf der klassischen Aussagenlogik und ihrer Definitionen\cite{klement2004propositional}.
Auf der Syntaxebene werden daher in EBNF die Produktionsregeln kleingeschrieben und die Terminale großgeschrieben.
Die Regel \(start\) wird zunächst als Startsymbol definiert. 
Von ihr lässt sich jedes Wort ableiten, welches die \(disjunction\) Regel erfüllt.
Jene Regel stellt eine aussagenlogische Oder-Verknüpfung beziehungsweise eine Disjunktion dar, 
welche wiederum aus der Regel für die Konjunktion oder für die Oder-Verknüpfung aus Disjunktion und Konjunktion ersetzt werden kann.
Die Konjunktionsregel \(conjunction\) kann entweder durch die \(expression\) Regel ersetzt werden oder 
durch eine Und-Verknüpfung der \(conjunction\) Regel und der \(expression\) Regel.
Die \(expression\) Regel stellt einen booleschen Ausdruck dar, welcher sich entweder mit der \(value \) Regel oder aus
einer Kombination von einer \(expression\) Regel, von einer \(operator\) Regel und von einer weiteren \(expression\) Regel ableiten lässt.

\begin{figure}[htb]
    \centering
    \includegraphics[width=1.0\textwidth]{images/syntax_baum.png}
    \caption{Generierter Syntaxbaum vom LALR-Parser}
~\label{fig:syntaxbaum}
\end{figure}

Bei der \(value\) Regel wird die Mehrdeutigkeit der Terminale definiert, für die alle primitiven Datentypen abgedeckt werden.
Die Terminale \(PROPERTY\_NAME\) wird hier hervorgehoben, da diese den regulären Ausdruck einer Eigenschaft eines Extraktionsobjekts abbilden soll.
Dabei ist dies als eine Verkettung durch weitere Objektnamen durch die Punktnotation definiert.
Für die \(operator\) Regel bestehen die Mehrdeutigkeiten aus der \(comp\_operator\) Regel, 
der \(ident\_operator\) Regel und der \(memb\_operator\) Regel. 
In der von mir aufgezählten Reihenfolge bilden die Vergleichsoperatoren, die Identitätsoperatoren und die Memberoperatoren ab. 
Die in Python existenten arithmetischen Operatoren, Zuweisungsoperatoren und bitweisen Operatoren werden 
für die aus der Grammatik erzeugten Sprache hingegen nicht definiert.
Des Weiteren wird das Klammern für die Ausdrücke nicht unterstützt, 
weswegen die Operatorrangfolge durch die Produktionsregeln der Grammatik fest vorgegeben ist.
Die restlichen Zeilen beinhalten die Importe der zusätzlichen Module für die Definition der Grammatik.
Abbildung~\ref{fig:syntaxbaum} zeigt ein Beispiel wie aus einem Regelausdruck ein Syntaxbaum durch den Parser kreiert werden kann.
Mittels der Transformer Klasse der Lark-Bibliothek wird als Zwischenschritt 
die Syntax der einzelnen Blätter und Knoten des Baums so angepasst, 
dass diese für die weitere Verarbeitung und Evaluation der Regelmaschine bereitstehen.


\subsection{Regelmaschine}\label{sec:ruleeng}

Nachdem unterdessen die Syntax und Grammatik der Regeln gegeben ist, wird die Regelmaschine als Modul entwickelt.
Dieses Modul besteht aus den Klassen für den Regelinterpreter und den Regelspeicher.
Während die Regelinterpreter-Klasse die konkrete Logik der Vorwärtsverkettung beinhaltet, 
kümmert sich die Regelspeicher-Klasse um das Bereitstellen und um die Speicherung der Regeln und Fakten.
Softwaretechnisch wird für diese Speicher-Klasse das Singleton Entwurfsmuster benutzt, 
womit die Regelmaschine nur eine Regelspeicher Instanz referenzieren kann.
Dadurch ist ein fester Zustand der Regel- und Faktenmenge gegeben.

Die Speicherung der Regeln und Fakten kann über den persistenten Ansatz der Datenbanken ermöglicht werden.
Jedoch erlaubt der Ansatz der Speicherung der Regeln und Fakten via einer Datei im Dateisystem den schnellen Zugriff 
und die schnelle Bearbeitung jener.
Des Weiteren stellt sich hier die Frage, in welchem Dateiformat die Regeln und Fakten festgehalten werden:
Ein gängiges Serialisierungsformat wie JSON eignet sich durch ein festgelegtes Schema zur Speicherung, 
da die Daten strukturiert aufbewahrt werden.
Demgegenüber existieren weitere Dateiformate, die eine geeignete Alternative darstellen.
Eines dieser Alternativen ist das Dateiformat YAML, welches auch als eine vereinfachte Auszeichnungssprache gilt. 
Jene Sprache bildet die Obermenge zu JSON ab und ermöglicht neben den auch in JSON definierten generischen Datentypen 
einen übersichtlicheren, aber auch komplexeren Ansatz für ein Serialisierungsformat. 
Die Sprache YAML besitzt im Vergleich zu JSON eine bessere Lesbarkeit der Syntax, 
ist kompakter gestaltet und bietet eine hohe Kompatibilität.~\cite{eriksson2011comparison}.
Die Syntax für mögliche Kommentare ist obendrein auch verfügbar, 
wodurch sich diese Auszeichnungssprache für Konfigurationsdateien eignet, 
das wiederum für die Definitionen der Regeln und Fakten abgebildet werden kann.

Die Serialisierungsformate unterscheiden sich letztendlich in der Zielsetzung:
JSON wird für den leichten Datenaustausch für Webanwendungen aufgrund der leichten Serialisierung 
und Übersetzung mit dem Parser benutzt.   
YAML dagegen hat eine erhöhte Komplexität für die Serialisierung und Übersetzung, 
aber ist für Dateien konzipiert, die von Menschen bearbeitet und genutzt werden. 

\begin{figure}[htb]
    \centering
    \includegraphics[width=.9\textwidth]{images/regelmaschine_sequenz.png}
    \caption{Durchlauf der Regelinterpretationen durch ein Request}
~\label{fig:regelmaschine_sequenz}
\end{figure}

Mit Hilfe einer Bibliothek für die Serialisierung und Übersetzung von YAML-Dateien werden ebendaher die Regeln und Fakten während der Laufzeit 
geladen, bearbeitet und gespeichert.
Dabei wird sich für den Ansatz entschieden, die Regeln und Fakten für jede einzelne Dokumentenart zu speichern und zu verwenden.
Während die booleschen Ausdrücke der Regeln als String dem Parser für die Grammatik übergeben werden, 
muss aus den Fakten nach dem Laden während der Laufzeit eine für die Evaluation angepasste Version generiert werden.
Hierfür werden durch eine Funktionalität die Fakten als eine Menge beziehungsweise als ein Dictionary erzeugt, 
die den Pfad zur Eigenschaft in der Punktnotation als Schlüssel besitzen, angegeben.
Ebenjene Konvertierung der Fakten ist für die Auswertung mit den Regeln im späteren Prozess geeignet.

Für den Ablauf wird, wie in Kapitel~\ref{sec:webserv} beschrieben, zunächst die Regelmaschine beim Starten des Systems instanziiert.
Diese wiederum instanziiert den Regelspeicher, welcher die Regeln und Fakten bereit hält, und den Regelinterpreter, 
welcher den Parser für die Grammatik Instanz erstellt.
Die Funktionalitäten des Interpreters werden aufgerufen, 
sobald die HTTP Anfrage bezüglich des Durchlaufens der Regelevaluation im WSGI ankommt.

Die Abbildung~\ref{fig:regelmaschine_sequenz} zeigt den Verlauf bei dieser Anfrage und den damit entstehenden Aufrufstapel.
So löst zunächst die Route-Anfrage die gekoppelte Funktion \(trigger\_run()\) der Flask Instanz aus, 
welche wiederum die \(run()\) Methode der Regelmaschine betätigt. 
Diese ruft dann die \(run()\) Methode des Regelinterpreters mit der Angabe, für welche Dokumentart der Vorgang stattfinden soll, auf.
Dort werden iterativ die aktuell gespeicherten Regeln mit Hilfe der Funktion \(interpret\_rule()\) zunächst evaluiert.
Diese Funktion übergibt den Ausdruck der einzelnen Regel dem Parser, 
mit welchem die Funktion \(eval\_expression\_tree()\) aufgerufen wird, um den Syntaxbaum zu erstellen.

Innerhalb dieser Methode wird rekursiv über den Syntaxbaum navigiert und die einzelnen booleschen Ausdrücke 
von den Blättern bis zur Wurzel aufgelöst.
Dieser boolescher Wert wird danach bis zum Methodenaufruf in der \(run()\) Funktion zurückgeben, 
sodass daraufhin die Regel bei einem positiven Ergebnis zu der Liste der erfüllten Regeln hinzugefügt wird. 
Ist die Schleife der Regelevaluation abgeschlossen, werden die Aktionen der erfüllten Regeln mit ihren Parametern 
in der \(invoke\_rule\_action()\) Funktion ausgeführt.
Hierbei werden die Aktionen abhängig des ausgewählten Dokumententyps aufgerufen.
Die Ergebnisse der aufgerufenen Funktionen werden dann mit den Regelnamen indexiert und bis zur WSGI Klasse,
in welcher der Wert als HTTP Antwort zurückgeschickt wird, gesammelt zurückzugeben.
Der Vorgang deckt somit den Hauptprozess des regelbasierten Ansatzes im Teilsystem.


\section{Entwicklungsumgebung}\label{sec:ide}

Damit die Funktionalitäten der Teilsysteme, die die unterschiedlichen Ansätze der künstlichen Intelligenz umsetzen, genutzt werden, 
bedarf es unter anderem die Steuerung jener über menschliche Eingaben.
So müssen sowohl die Möglichkeiten der Konfiguration der Regeln und Fakten 
für die Benutzer:innen als auch das Starten des maschinellen Lernens über eine Benutzungsschnittstelle gegeben sein.
Wie in Kapitel~\ref{sec:ruleeng} bereits erwähnt, muss eine Umgebung zur Verfügung gestellt werden, 
die das Bearbeiten der Regel- und Faktendefinitionen erlaubt.

Die Hauptfunktionalitäten wie das Starten, das Überprüfen der Änderungen und das Pausieren sollen direkt erreichbar sein, 
weswegen sich die Positionierung dieser in der Navigationsleiste eignet.
Des Weiteren soll der Dokumententyp für den Extraktionsprozess auswählbar sein 
und die Dokumente für den ausgewählten Dokumententyp in dieser Umgebung als eine Seitenansicht angezeigt werden.
Auch sind die Ergebnisse der Extraktionsprozesse für die Benutzer:innen für den Abschluss und für die Entwicklung relevant, 
weswegen eine Anzeige dafür erstellt wird.
Das Layout der Benutzungsoberfläche orientiert sich hierbei an bekannten Entwicklungsumgebungen 
wie Visual Studio~\cite{microsoftVisualStudioIDE} oder IntelliJ~\cite{jetbrainsIntelliJIDEACapable}.
Durch dieses Layout wird eine intuitivere Interaktion geboten.

Da die Schnittstellen über das Netzwerk der Client-Server-Architektur gegeben sind 
und die Priorität auf der Zugänglichkeit und Integration in das System liegt,
habe ich mich für die Umsetzung der Entwicklungsumgebung als Webanwendung entschieden.
Für die graphische Schnittstelle der Benutzer:innen des Systems wird daher die UI Bibliothek Vue benutzt, 
die für eine kleine prototypische Umsetzung mit anderen Bibliotheken wie React bei dem Kompilieren und Laden zeitlich mithalten kann~\cite{pikkanen2021react}.
Bei Vue habe ich mich für die dritte Hauptversion mit der Composition Application Interface (API), 
die eine reaktive Programmierung für das ermöglicht, 
und für die Nutzung von dem Dateiformat Single File Components (SFC) entschieden~\cite{vueCompositionAPIFAQ}. 
Des Weiteren wird das Build Tool Vite benutzt, die mit Hilfe eines Optimierungsansatzes 
die Kompilierung und die Übersetzung des Codes auf das Frontend umlagert~\cite{viteVite}.

Für Vite wird daher eine Konfigurationsdatei angelegt, 
die Einträge über den Entwicklungsserver und die verschiedenen Worker für den im späteren Verlauf vorgestellten Editor besitzt.
Dazu nutzt man konventionell der Node Package Manager (npm), 
um weitere Abhängigkeiten für das Frontend zu installieren~\cite{npmNpm}.
Weitere Eigenschaften des Frontends sind die Nutzung von der Programmiersprache TypeScript, 
mit welcher die Typsicherheit für die dynamische Skriptsprache gewährleistet wird~\cite{bierman2014understanding}, 
und dem Entwicklungswerkzeug PostCSS.
 Mit PostCSS wird eine detaillierte und übersichtliche Definition für Style Selektoren ermöglicht~\cite{postcssPostCSSToolTransforming}.
Als Einstiegspunkt wird klassischerweise eine Hyper Text Markup Language (HTML) Datei verwendet, 
die das Skript zum Instanziieren der Vue Anwendung auslöst. 
An jener Stelle werden wiederum die weiteren über npm bereits installierten Bibliotheken 
als Plugin in die Vue Anwendung installiert.

\begin{figure}[htb]
    \centering
    \frame{\includegraphics[width=1.0\textwidth]{images/ide_view.png}}
    \caption{Hauptansicht der Entwicklungsumgebung}
~\label{fig:ide_view}
\end{figure}

Für die Vue Anwendung wird die Struktur modular gehalten, weswegen verschiedene Vue Komponente definiert werden.
Diese werden in die Hauptkomponenten und weiteren Unterkomponenten hierarchisiert.
So instanziiert die Hauptkomponente die Komponenten für die Bereiche der Hauptansicht.
Darunter fallen die Bereiche für die Navigationsleiste, für den Editor, für die PDF-Ansicht und für den Output 
für diese die jeweiligen Komponenten als Modul definiert sind.
In Abbildung~\ref{fig:ide_view} die im Rahmen meiner Thesis entwickelten prototypische Version der Entwicklungsumgebung angezeigt.
Hier sind die beschriebenen Bereiche zu sehen, die in eigene Module unterteilt wurden.
Zwischen den Hauptbereichen befinden sich Trennbalken, die durch eine weitere Bibliothek implementiert wurde 
und mit deren Hilfe die Größen der verschiedenen Bereiche verändert werden können.
Eine wichtige Eigenschaft ist die Nutzung eines State-Management-Systems unter Zuhilfenahme der Bibliothek \(Pinia\)~\cite{piniaPinia}.
Dies ermöglicht das Zwischenspeichern der einzelnen Zustände der Komponenten, 
die dem Server mitgeteilt werden kann, oder das Erhalten des letzten Zustands für das Neuladen der Webanwendung.
So können beim initialen Aufruf die letzten Werte geladen, womit diese Anwendung offline-fähig dargestellt werden kann.

\subsection{Editor}\label{sec:editor}

Die Konfiguration der Regeln und Fakten bedarf eine UI Schnittstelle für das Erstellen und Bearbeiten dieser.
Dies wird gängigerweise in Form eines Texteditors ermöglicht.
Im Webbereich stehen für die Integration eines Texteditors mindestens mehrere Bibliotheken als Kandidaten zu Verfügung.
Im Rahmen des Projekts wurde sich für die Nutzung des Monaco-Editor entschieden, 
welches die Kernkomponente des Editors Visual Studio Code~\cite{microsoftVisualStudioCode} darstellt 
und unabhängig von ihr als Open-Source Projekt bereitgestellt wird.

Die folgenden Eigenschaften führten zu der Auswahl:
Der Monaco-Editor bietet viele Funktionalitäten für eine angenehme Bearbeitung, darunter die Autovervollständigung 
durch IntelliSense, Textsuche und Code Linting.
Die Bibliothek wird durch eine weite Community gewartet, 
besitzt allerdings einen hohen Umfang des Software-Bundlings und 
erlaubt auch keine Lazy Loading der Funktionalitäten als Entwurfsmuster.
Des Weiteren müssen die Worker für die einzeln genutzten Sprachen konfiguriert werden.
Diese Worker werden in der zuständigen Vue Komponente definiert.

Danach wird nach der Reihenfolge das JSON Schema vom Server angefragt und in den Editor geladen, 
welches zusätzlich für IntelliSense und das Code Linting definiert wurde~\cite{jsonJSONSchema}.
Im weiteren Code Verlauf werden weitere Einstellungen, wie die gewählte Sprache, für den Monaco-Editor angegeben, 
sodass dieser instanziiert werden kann.
So wird die Sprache für das Highlighting auf YAML gestellt, welches als Dateiformat für die Regeln und Fakten gewählt wurde.
Hier wird zunächst ein Request zum Laden der Einträge des ausgewählten Dokumenttyps ausgeführt.

Als Callback wird daraufhin festgelegt, dass nach jeder Textänderung wenige Sekunden später ein Request 
für die Speicherung dieser Änderungen zu dem jeweiligen Dokumententyp geschickt wird. 
Somit werden die Zustände der Konfiguration auf der Server Seite festgehalten.
Auf der Benutzungsoberfläche wird dann Texteditor angezeigt, der auch die Navigation über Tabs unterstützt.
So können die Regeln und Fakten in einem Tab oder in zwei getrennte Tabs gehalten werden, 
was wiederum eine übersichtlichere Aufteilung verschafft.


\subsection{Dokumentenanzeige}\label{sec:docan}

Damit die Entwicklung und das Anpassen der Regeln und Fakten angenehmer verläuft, 
bedarf es einer Anzeige der Dokumente als Orientierungspunkt.
Diese umgesetzte Anzeige erlaubt es, eine einzelne Seitenansicht der Dokumente des ausgewählten Dokumenttyps darzustellen.
Hierfür steht ein Selektor bereit, welcher eine Auswahl aus allen Dokumenten des Dokumenttyps gewährleistet 
und einen schnellen Wechsel anbietet. 
Daneben ist das Vor- und Zurückblättern mit zwei Knöpfen beziehungsweise das Umblättern mit Hilfe einer Seitenanzahl gegeben.

Damit die Positionierung des zu extrahierenden Bereichs konfiguriert werden kann, 
helfen die Koordinaten, die abhängig des Mauszeigers als Referenzpunkt angezeigt werden.
Dies wurde dynamisch mit der Hover-Funktion umgesetzt, womit die Benutzer:innen direkt die X- und Y-Koordinaten erhalten können.
Außerdem soll die Anzeige im Zusammenspiel mit dem Editor es ermöglichen, 
die Bounding Box eines Bereichs durch das Ziehen mit der Maus 
auf der ausgewählten Seite zu generieren und diese in eine ausgewählte Stelle im Editor einzusetzen.

\subsection{Output}\label{sec:out}

Nachdem die Benutzer:innen ihre Konfiguration festgelegt und den Durchlauf gestartet haben, 
bedarf es für die Interaktion eine Rückgabe des Systems.
Hierfür wurde sich für ein Ausgabebereich entschieden, welches zwei Fenster besitzt:
Das erste Fenster stellt die Ergebnisse in einer Anzeige für JSON Dateien dar 
und ermöglicht damit eine genauere Betrachtung für die Benutzer:innen. 
Hierbei werden sowohl die konkreten Ergebnisse des Durchlaufs der Regeln als 
auch die Ergebnisse der Auswirkungen der Änderungen angezeigt.

Beim zweiten Fenster werden Informationen, die abseits von Ergebnisse, wie Zustände des Systems oder Fehlermeldungen.
Dieses Fenster soll es ermöglichen, einen aktuellen Zustand und eine Rückmeldung zu präsentieren, 
womit die Mensch-Computer-Interaktion verbessert wird und die Nutzungsbarriere zum System niedriger wird.
Unter den Zuständen zählen die erfolgreiche Inbetriebnahme, Hinweise zur Nutzung und Fehlermeldungen, 
die im System auftreten potentiell können.

\section{Machine Learning}\label{sec:ml}

Während die Teilsysteme der symbolischen künstlichen Intelligenz und der Entwicklungsumgebung implementiert sind, 
bedarf es nun die Konzeption und Umsetzung des Machine-Learning-Ansatz.
Für die gleiche Problematik, die bereits von dem Teilsystem der symbolischen KI thematisiert und verarbeitet wurde, 
soll mit dem Ansatz ein weiterer Lösungsweg zur Bewältigung dieser gewählt werden.

Für das maschinelle Lernen existieren unter anderem die Arten des überwachten Lernens (supervised Learning) 
und des nicht überwachten Lernens (unsupervised Learning).
Beide haben ihre Einsatzgebiete, so auch im Falle der Dokumentenextraktion.
Hierbei zählen in dem Bereich der Extraktion verschiedene Dokumentarten wie Text- oder HTML Dokumenten hinzu, 
bei welchem die letztere genannte Art 
präzise Ergebnisse über die Methoden des überwachten Lernens verschafft~\cite{gopinath2018automatic}.
Nicht überwachtes Lernen hat im Gegensatz bei Dokumenten, die kein Labeling besitzen, einen höheren Wirksamkeit.

Über beide Arten des maschinellen Lernens können Daten strukturiert und modelliert werden.
Doch existieren neben den weit verbreiteten zwei Kategorien auch weitere Bereiche im Machine Learning.
Das semi-überwachtes Lernen verhilft das Labeling durch weitere Annahmen der Struktur, 
durch semi-überwachte Klassifikation und durch eingeschränktes Clustering zum Durchbruch~\cite{zhu2009introduction}.
Das schwache überwachte Lernen baut als Technik die Datenmodelle über neu generierte Daten.
Dies ist für den Fall geeignet, wenn man mit neuen Daten aus Dokumenten arbeiten will.
Für diesen Ansatz ergibt die Bibliothek Fonduer mit der Erstellung einer Pipeline Struktur einen neue Lösungsmöglichkeit.

\subsection{Pipeline}\label{sec:pipe}

Fonduer setzt den Machine Learning Ansatz mit Hilfe der Kategorie 
eines abgeschwächten überwachten Lernen (weak supervised learning) um.
Genauer beschrieben, wird hier die Idee des Knowledge Base Construction (KBC), 
also des Aufbaus einer Wissensbasis über stark formatierte Daten verfolgt.
Diese Wissensbasis wird über eine Datenbank abgebildet, die die Daten strukturiert bereithält.
Daher besteht für die Nutzung dieser Bibliothek eine Abhängigkeit zu dem Datenbanksystem PostgreSQL~\cite{elmasriFundamentalsDatabaseSystems2007}. 
Des Weiteren wird die Bibliothek spaCy genutzt, weswegen die Sprachmodelle für die deutsche Sprache 
in dem Dockerfile des Teilsystems mit geladen und verlinkt werden.

Damit dies auch für vielschichtige Daten geht, wird hier ein spezielles Datenmodell benutzt, 
welches die Dokumente und ihre Besonderheiten darstellt.
So fallen verschieden Formate von Dokumenten wie XML, HTML und auch PDF unter dem Begriff der reichlich formatierten Daten.
Hier sollen die textlichen, visuellen und strukturellen Eigenschaften der Dokumente mit ihrer semantischen Informationen 
verbunden werden, um sie für die weitere Verarbeitung zu nutzen. 
Diese Eigenschaft wird als Multimodalität bezeichnet und taucht in der Prozesskette an verschiedenen Stellen auf.

Die Eingabe für das Teilsystem findet in Form der Dokumente statt.
Auch für dieses Teilsystem wird sich auf die Dokumente bezogen, die im Rahmen der Thesis betrachtet werden.
Hierfür werden, wie auch im anderen Extraktionssystem, die PDF-Dokumente in ihre drei Dokumenttypen unterteilt.
Die Menge der Dokumente für die jeweilige Art unterscheiden sich in ihre Dimension:
Für die Dokumente der Care Garantieaufträge ist die Anzahl der Elemente dreistellig, für die Formel1 Rechnung vierstellig 
und für die KFZ300 Rechnung fünfstellig. 
Dies soll eine Relation der Verarbeitung ermöglichen und damit einen zeitlichen Zeitverlauf darstellen.

Der Durchlauf von den Eingabedaten zu der Erstellung des Machine-Learning Modells wird als Pipeline bezeichnet.
In Abbildung~\ref{fig:fpipeline} wird die komplette Pipeline für die Verarbeitung des maschinellen Lernens dargestellt.
So sind die einzelnen Phasen in zwei Kategorien aufgeteilt zu sehen. 
Die erste Hälfte der Phasen ist statisch vorgegeben und kann strukturell nicht verändert werden.
Bei der zweiten Hälfte sind Eingriffe der Benutzer:innen beziehungsweise 
bei den Entwickler:innen möglich um die Datenmodelle zu präzisieren.
Insgesamt sind die fünf Phasen relevant, die in der Hauptfunktionalität im Teilsystem umgesetzt wurden:
In der ersten Phase wird zunächst die KBC über ein spezielles Schema und 
daraus folgenden über die iterative Erstellung eines Eingabekorpus initialisiert.
Dafür wird die Hostadresse der benötigten Datenbank angegeben und diese einer Meta-Objekt, welcher die Brücke zur Persistenz darstellt, übergeben, 
um die Datenbank parallel zu starten.

\begin{figure}[htb]
    \centering
    \includegraphics[width=1.0\textwidth]{images/fonduer_pipeline.png}
    \caption{Fonduer Pipeline für das maschinelle Lernen}
~\label{fig:fpipeline}
\end{figure}

Die multimodale Datenmodellierung wird in dieser Phase durch das parallele Parsen der Dokumente ermöglicht, 
welches inhaltliche, strukturelle, tabellarische und visuelle Informationen bereit hält.
Für das Parsing findet hierbei die Nutzung der HOCRDocPreprocessor-Klasse statt, 
welches den Pfad der zu übersetzenden Dateien bei Instanziierung mitgegeben wird.
Als Zwischenschritt werden die PDF-Dokumente als HOCR Dateien, einem Optical Character Recognition (OCR) Datenformat, 
welches auf HTML aufbaut, mittels einer internen Klasse übersetzt.
Ein weiterer Weg wäre dies über den HTMLPreprocessor umzusetzen, was aber bei digital verarbeiteten Dateien nicht von der 
Dokumentation empfohlen wird.

Als nächstes wird der Korpus-Parser benutzt der mit Hilfe der Bibliothek spaCy die Dokumente in einzelne Sätze und Tokens aufteilt 
und über weitere Parametrisierung die Inklusion der strukturellen und sprachlichen Informationen angegeben wird.
In der zweiten Phase werden Mentions, welche die potentiellen Kandidaten für die Datenextraktion darstellen, definiert.
In dem Anwendungsfall des Projekts wird für jedes Feld, die sich in den Aufträgen und den Rechnungen befinden 
und extrahiert werden sollen, eine Mentions festgelegt.
Beispiele hierfür wären, wie aus dem Teilsystem der symbolischen KI bereits bekannt, die Jobs, die Reparaturzeiten, 
die Fahrzeugmodelle und die Jobbeschreibungen.

Damit diese Mentions auch vollwertige Kandidaten des Extraktionsziels werden, bedarf es die Definitionen von Matchers, 
die über eine Filterangabe eine präzise Erkennung ergeben sollen,
Die Matchers gibt es in verschiedenen Variationen, in diesem Anwendungsfall 
wird sich für die Filterung über reguläre Ausdrücke entschieden, da diese bei der symbolischen KI ihre Anwendungen fanden.
So wird zu jedem Mention Objekt ein Matcher Objekt entwickelt.
Dazu kommen noch jeweils Definitionen für die N-Gramm Darstellung der Mention-Objekte um so die Größe der Fragmente festzulegen.

Damit die Extraction der Mentions mit all diesen zusätzlichen Einschränkungen stattfinden kann, 
wird ein MentionExtractor-Objekt definiert, 
welches die Mentions und deren Matchers und N-Gramms mitgegeben wird, 
um mit dieser Kombination an Informationen die erwünschten Daten zu extrahieren.
Nachdem dieses Informationen auf das MentionExtractor-Objekt angewendet werden, 
wird das Schema der Kandidaten für die Extraktion angegeben.
Hierfür definieren für jede Dokumentenart eine eigene Unterklasse der Candidate-Klassen,
So definieren wir ein Warranty-Objekt für die Care Aufträge, ein Invoice-Objekt für die Formel1 Rechnungen 
und ein Invoice-Objekt für die KFZ3000 Rechnungen.
Diese bekommen wie bei der Mentions-Klasse, die selben Matcher übergeben.

Dazu werden Throttlers, Funktionen die als weitere starke Filter fungieren, 
definiert, um die Genauigkeit der validen Extraktionskandidaten zu ermöglichen. 
In diesem Anwendungsfall definieren wir Throttlers, die sich mit den Ausrichtungen der verschiedenen Einträgen beschäftigen.
So wird zum Beispiel überprüft, ob die Kandidaten für die Reparaturzeiten, 
der Beschreibung und der Menge horizontal zueinander ausgerichtet sind.
Nachdem dies festgelegt ist, werden diese Paradigmen in einem CandidateExtractor-Objekt kombiniert und angewendet.
Für die Anwendung werden die Dokumente in Trainings- und Testsätze unterteilt.

In der dritten Phasen werden den Kandidaten der Trainings- und Testsätzen durch multimodale Featurisierung Merkmale annotiert.
Hierfür wird ein Featurizer-Objekt definiert, welches die Verbindung zur Datenbank 
und den Kandidaten der jeweiligen Dokumentenart übergeben bekommt.
Für die vierte Phase findet das schwache überwachte Lernen und die Klassifikation über Labeling-Funktionen.
Dafür werden diese Funktionen, welche weitere Bedingungen für die Kandidaten stellen und 
die es zu erfüllen gilt, definiert, um als wahrhaftiger Kandidat weiterverarbeitet zu werden.
Die Labeling Funktionalitäten werden hierbei durch die Bibliothek Snorkel zur Verfügung gestellt, 
welche das Labeling der Trainingssätze durch das schwache maschinelle Lernen schneller und günstiger der Ressourcenkosten gestaltet~\cite{Ratner_2017}.

Daher reichen zunächst einfache Funktionsmuster, die die tabellarischen Eigenschaften der Felder, 
wie zum Beispiel der Fahrzeugmodellnamen, des Kfz-Kennzeichens und des zuletzt abgelesen Kilometerzählerstands, überprüfen.
Stehen die Labeling Funktionen, können diese auf das Labeler-Objekt angewendet werden.
Hierbei können weitere Statistiken des Vorgangs als Feedback mitgegeben werden.
Ist das Training abgeschlossen, werden als Nächstes in der Pipeline die Labeling-Funktionen modelliert, 
sodass Konflikte betrachtet werden können.
Mit Wahrscheinlichkeitsberechnungen werden dann weitere Labels auf die Trainingssätze angewendet, 
sodass danach ein generatives Modell angelernt ist.

Dies ermöglicht dem System und der Entwicklung iterativ die Labeling Funktionen für eine höhere Genauigkeit zu verbessern.
Hierfür müssen die Konflikte und die Überlappungen programmatisch weiter überprüft werden.
Für das weitere Training der Datensätze wird das discriminative Modell durch Regression antrainiert.
Dafür benutzt Fonduer das Machine-Learning Framework Emmental, welches die Funktionalitäten für die Modellierung bereitstellt\cite{wu2019emmental}.
In diesem Prozess werden weitere Tokens und Rauschproben generiert, damit das Modelltraining unterstützt wird.
Zum Schluss können die Testsätze mit ihrem Labeling genutzt werden, um die Vorhersage durch das Training der KI zu evaluieren.

\subsection{Webserver}\label{sec:webserv2}

Damit die Steuerung der künstliche Intelligenz des maschinellen Lernens und ihre Funktionalitäten gegeben ist, 
wird die Kommunikation der Teilsysteme benötigt.
Diese Kommunikation findet über das interne Netzwerk statt und ermöglicht damit auch die Interaktion der Docker Container Umgebung.
Damit diese nötigen Eigenschaften für das Teilsystem erfüllt werden können, 
wird der selbe Ansatz des als erstes vorgestelltes Teilsystems umgesetzt.
Das bedeutet, dass sich für die Nutzung der WSGI Schnittstelle entschieden wurde, 
was auch die Verwendung einer Flask Anwendung als Webserver zur Folge hat. 

So wird auch eine Factory-Methode für verschiedene Anwendungsfälle, 
wie unter anderem für die Continuous Integration, bereitgestellt.
Die Entwicklungsumgebung erlaubt hierbei das Starten der Verarbeitung der Trainingsdaten,
weswegen sich hier die die Funktionalität nur auf den Durchlauf mittels des Machine-Learning-Ansatzes reduziert.


\chapter{Ergebnisse}\label{chap:erg}

Das Projekt hat verschiedene Merkmale der Problematik hervorgehoben.
So ist aufgefallen, dass die Dokumente, auf die sich im Rahmen der Thesis eingeschränkt wurde, viele verschiedene Formen besitzen 
beziehungsweise haben können.
Eine Anpassung für die Definitionen der Formen muss in der symbolischen künstlichen Intelligenz 
durch eine Kategorisierung in Dokumentarten vorgenommen werden.
Außerdem fiel auf dass, sowohl die Definitionen von Tabellen in elektronisch verarbeiteten Dokumenten, 
als auch die Erkennung dieser sich als umständlich erweist.

So bedarf es zunächst Analysen der Struktur und des Kontexts um eine angemessene Verarbeitung zu ermöglichen.
Das System, dass im Rahmen dieser Thesis entwickelt wurde, erwies sich als Prototypen umsetzbar.
Die Struktur für die Teilsysteme wurde durch eine container-getriebene Entwicklung modular gehalten 
und sorgte durch die Architektur eine gemeinsame Kommunikationsebene für diese Komponenten.
So konnte ein Projekt aufgesetzt werden, welches die neuesten Technologien in den Bereichen, in denen diese agieren, 
nutzen um die erwünschten Extraktionsziele zu erhalten oder eine Annäherung mindestens zu erreichen. 
Texte und Tabellen können über Regel- und Faktendefinitionen extrahiert werden 
und übersichtlich in einer Webanwendung dargestellt werden.
Die Entwicklung einer Grammatik 



\section{Testing}\label{sec:test}

Für die prototypische und zukünftige Entwicklung des Projekts ist das Testing für die weitere Benutzung relevant. 
Hauptsächlich hierfür verantwortlich soll die Continuous Integration Pipeline dafür sein.
Zur Grunde liegen hier die Images der Teilsysteme, die als Dockerfile beschrieben sind 
und in der Container Registry vom unternehmensinternen Repository gespeichert werden.
Über verschiedene Stages sollen dann die Hauptaspekte der Teilsysteme und dann in das System integriert werden.

Hier werden diese Komponente auf verschieden Eigenschaften überprüft.


\chapter{Konklusion und Ausblick}\label{chap:kua}

In dieser Arbeit ging es darum, Informationen aus digital verarbeiteten PDF-Dokumenten zu extrahieren und 
aus diesen neuen Daten zu generieren. 
Die Dokumente, um die es in dieser Arbeit ging, stammten von Kund:innen von Ilexius GmbH. 
Jene Kund:innen sind unter anderem Beschäftigte von Autohäusern und deren Werkstätten.  
Mit meiner Systementwicklung, die ich in dieser Thesis demonstriert habe, 
werden Informationen aus dem Endprodukt aller Arbeitsschritte der Kund:innen für die Entwickelnden besser zugänglich gemacht, 
indem die Daten leichter extrahiert werden, sodass ilexius diese Daten für weitere Arbeitsschritte im ERP-System bereitgestellt werden kann.
Für meine Entwicklung habe ich das Data-Mining im Bereich der Dokumentenextraktion genutzt, das dabei hilft, 
laufende Prozesse detaillierter zu steuern und zu überwachen. 

Neben dem Data-Mining nutzte ich zudem den Ansatz einer symbolischen künstlichen Intelligenz. 
Diese KI ermöglichte mir während meines Arbeitsprozesses in jedem Schritt eine Feinschleifung meiner bisherigen Arbeit. 
Durch diese KI kann sich jede/r Benutzer/in meines hier entwickelten Systems von groben zu detaillierten 
und angepassten Definitionen über ein eigenständiges Feedback des Systems annähern. 
Im Verlauf meiner Arbeit übernahm diese Annäherung das Machine Learning.
Bevor meine symbolische KI als auch das Machine Learning starten konnte, 
habe ich zu Beginn die Funktionalitäten der Bibliotheken genutzt, um durch Ergebnisse der Datenansätze die Erfolgsquote zu überprüfen. 
Danach 
Anschließend stellte ich die Benutzungsoberfläche fertig, sodass ich diese mit der symbolischen KI verbinden konnte. 
Nebstdem konzeptionierte und entwickelte ich die zu dem Machine Learning gehörende Subsysteme. 
Dies tat ich in einem iterativen Ansatz, sodass das Training des Machine Learnings baldmöglichst beginnen konnte. 
Im Anschluss daran fing ich mit dem Software-Testing an. 
Dadurch konnte die Ausführung aller meiner noch folgenden Prozesse gewährleitet werden. 
Als letzte Vorbereitung passte ich das grafische Tool weiter an, dass sich dieses mit den Systemen verbindet, 
wodurch die Ausführbarkeit der Funktionalität überprüft werden kann.

Meine Ergebnisse dieser Arbeit…
	Komplikationen nennen	
    Was kannst du ich Zukunft aus den Komplikationen lernen, welche Schritte bedarf es für eine Verbesserung
	Danach wieder zu einem positiven Feedback kommen, schöner Abschlusssatz, wie hilfreich deine Arbeit war


    Im Rahmen der Abschlussarbeit wurde ein System konzipiert und umgesetzt, 
    welches eine Extraktion von Dokumenten über zwei Ansätze ermöglicht.
    
    
    Über die Teilsysteme wurden technische Funktionalitäten eingeführt werden um dem Kapitän die Webanwendung als PWA anzubieten.
    
    Insgesamt eignen sich Webanwendungen um Ressourcen übersichtlicher darzustellen und mobil zu beobachten.
    Auch können Offline-Zustände dem Benutzer trotzdem erlauben Aspekte der Anwendung weiter zu benutzen.
    Aus den Konzepten und der Umsetzung ergeben sich viele weitere Ideen und Ansätze, 
    die realisierbar sind und implementiert werden können. 



\newpage

% Listen wenn überhaupt ans Ende und nicht an den Anfang.
% Meist ist das aber unnötig.
%\listoffigures % Liste der Abbildungen 
%\listoftables % Liste der Tabellen
% \newpage

\bibliography{thesis}
\bibliography{online}
\bibliographystyle{plain} % Literaturverzeichnis
\begin{btSect}{thesis} % mit bibtopic Quellen trennen
\section*{Literaturverzeichnis}
\btPrintCited{}
\end{btSect}
\begin{btSect}{online}
\section*{Online-Quellen}
\btPrintCited{}
\end{btSect}
% dann mit "bibtex thesis1" und "bibtex thesis2" arbeiten

\end{document}
;;; Local Variables:
;;; ispell-local-dictionary: `de_DE-neu'
;;; End:
